\section{Manacher算法}

这里我们将只描述算法中寻找所有奇数长度子回文串的情况,即只计算 $d_1[]$ ;寻找所有偶数长度子回文串的算法(即计算数组 $d_2[]$ )将只需对奇数情况下的算法进行一些小修改。

为了快速计算,我们维护已找到的最靠右的子回文串的 \textbf{边界 $(l, r)$} (即具有最大 $r$ 值的回文串,其中 $l$ 和 $r$ 分别为该回文串左右边界的位置)。初始时,我们置 $l = 0$ 和 $r = -1$ ( \textit{-1} 需区别于倒序索引位置,这里可为任意负数,仅为了循环初始时方便) 。

现在假设我们要对下一个 $i$ 计算 $d_1[i]$ ,而之前所有 $d_1[]$ 中的值已计算完毕。我们将通过下列方式计算:

\begin{itemize}
    \item \textit{如果 $i$ 位于当前子回文串之外,即 $i > r$ ,那么我们调用朴素算法。}\\
    因此我们将连续地增加 $d_1[i]$ ,同时在每一步中检查当前的子串 $[i - d_1[i] \dots i + d_1[i]]$ ($d_1[i]$ 表示半径长度,下同)是否为一个回文串。如果我们找到了第一处对应字符不同,又或者碰到了 $s$ 的边界,则算法停止。在两种情况下我们均已计算完 $d_1[i]$ 。此后,仍需记得更新 $(l, r)$ 。
    \item \textit{现在考虑 $i \le r$ 的情况。}\\
    我们将尝试从已计算过的 $d_1[]$ 的值中获取一些信息。首先在子回文串 $(l, r)$ 中反转位置 $i$ ,即我们得到 $j = l + (r - i)$ 。现在来考察值 $d_1[j]$ 。因为位置 $j$ 同位置 $i$ 对称,我们 \textbf{几乎总是} 可以置 $d_1[i] = d_1[j]$ 。该想法的图示如下(可认为以 $j$ 为中心的回文串被“拷贝”至以 $i$ 为中心的位置上):

    $$
    \ldots\
    \overbrace{
        s_l\ \ldots\
        \underbrace{
            s_{j-d_1[j]+1}\ \ldots\ s_j\ \ldots\ s_{j+d_1[j]-1}
        }_\text{palindrome}\
        \ldots\
        \underbrace{
            s_{i-d_1[j]+1}\ \ldots\ s_i\ \ldots\ s_{i+d_1[j]-1}
        }_\text{palindrome}\
        \ldots\ s_r
    }^\text{palindrome}\
    \ldots
    $$
    然而有一个 **棘手的情况** 需要被正确处理:当“内部”的回文串到达“外部”回文串的边界时,即 $j - d_1[j] + 1 \le l$ (或者等价的说, $i + d_1[j] - 1 \ge r$ )。因为在“外部”回文串范围以外的对称性没有保证,因此直接置 $d_1[i] = d_1[j]$ 将是不正确的:我们没有足够的信息来断言在位置 $i$ 的回文串具有同样的长度。

    实际上,为了正确处理这种情况,我们应该“截断”回文串的长度,即置 $d_1[i] = r - i$ 。之后我们将运行朴素算法以尝试尽可能增加 $d_1[i]$ 的值。

    该种情况的图示如下(以 $j$ 为中心的回文串已经被截断以落在“外部”回文串内):

    $$
    \ldots\
    \overbrace{
        \underbrace{
            s_l\ \ldots\ s_j\ \ldots\ s_{j+(j-l)}
        }_\text{palindrome}\
        \ldots\
        \underbrace{
            s_{i-(r-i)}\ \ldots\ s_i\ \ldots\ s_r
        }_\text{palindrome}
    }^\text{palindrome}\
    \underbrace{
        \ldots \ldots \ldots \ldots \ldots
    }_\text{try moving here}
    $$

    该图示显示出,尽管以 $j$ 为中心的回文串可能更长,以致于超出“外部”回文串,但在位置 $i$ ,我们只能利用其完全落在“外部”回文串内的部分。然而位置 $i$ 的答案可能比这个值更大,因此接下来我们将运行朴素算法来尝试将其扩展至“外部”回文串之外,也即标识为 "try moving here" 的区域。
\end{itemize}\par

最后,仍有必要提醒的是,我们应当记得在计算完每个 $d_1[i]$ 后更新值 $(l, r)$ 。

同时,再让我们重复一遍:计算偶数长度回文串数组 $d_2[]$ 的算法同上述计算奇数长度回文串数组 $d_1[]$ 的算法十分类似。
\begin{lstlisting}
vector<int> d1(n);
for (int i = 0, l = 0, r = -1; i < n; i++) {
  int k = (i > r) ? 1 : min(d1[l + r - i], r - i);
  while (0 <= i - k && i + k < n && s[i - k] == s[i + k]) {
    k++;
  }
  d1[i] = k--;
  if (i + k > r) {
    l = i - k;
    r = i + k;
  }
}
vector<int> d2(n);
for (int i = 0, l = 0, r = -1; i < n; i++) {
  int k = (i > r) ? 0 : min(d2[l + r - i + 1], r - i + 1);
  while (0 <= i - k - 1 && i + k < n && s[i - k - 1] == s[i + k]) {
    k++;
  }
  d2[i] = k--;
  if (i + k > r) {
    l = i - k - 1;
    r = i + k;
  }
}
\end{lstlisting}
