\section{后缀数组}
\subsection{倍增+sort $O(n log^2n$)}
\begin{lstlisting}
const int N= 1e6 + 8;
int sa[N], rk[N << 1], oldrk[N << 1];
int w;
void sa_sort(const char* s){
    int n = strlen(s + 1);
    for(int i = 1; i <= n; i++) sa[i] = i, rk[i] = s[i];
    for(w = 1; w < n; w <<= 1){
        sort(sa + 1, sa + n + 1, [](int x, int y){
           return rk[x] == rk[y] ? rk[x + w] < rk[y + w] : rk[x] < rk[y];
        });
        memcpy(oldrk, rk, sizeof(oldrk));
        for(int p = 0, i = 1; i <= n; i++){
            if(oldrk[sa[i]] == oldrk[sa[i - 1]] && oldrk[sa[i] + w] == oldrk[sa[i - 1] + w]){
                rk[sa[i]] = p;
            }
            else rk[sa[i]] = ++p;
        }
    }
    for(int i = 1; i <= n; i++) printf(i == n ? "%d\n":"%d ", sa[i]);
}
\end{lstlisting}
\subsection{计数排序版本 $O(n\log n)$}
\begin{lstlisting}
// 多组数据时要注意memset和memecpy复杂度会假!!!
const int N = 1E6 + 7;
const int S = 128;
int rk[N], oldrk[N << 1], sa[N], cnt[N], id[N], px[N];
int height[N];
// px[i] = rk[id[i]]
bool cmp(int x, int y, int w){
    return oldrk[x] == oldrk[y] && oldrk[x + w] == oldrk[y + w];
}
void get_sa(const char* s){
    int i, w, p, m = S;
    int n = strlen(s + 1);
    for(i = 1; i <= n; i++) ++cnt[rk[i] = s[i]];
    for(i = 1; i <= m; i++) cnt[i] += cnt[i - 1];
    for(i = n; i >= 1; i--) sa[cnt[rk[i]]--] = i;
    for(w = 1;; w <<= 1, m = p){
        for(p = 0, i = n; i > n - w; i--) id[++p] = i;
        for(i = 1; i <= n; i++) if(sa[i] > w) id[++p] = sa[i] - w;
        memset(cnt, 0, sizeof(cnt));
        for(i = 1; i <= n; i++) ++cnt[px[i] = rk[id[i]]];
        for(i = 1; i <= m; i++) cnt[i] += cnt[i - 1];
        for(i = n; i >= 1; i--) sa[cnt[px[i]]--] = id[i];
        memcpy(oldrk, rk, sizeof(rk));
        for(p = 0, i = 1; i <= n; i++) rk[sa[i]] = cmp(sa[i], sa[i - 1], w) ? p : ++p;
        if(p == n){
            for(i = 1; i <= n; i++) sa[rk[i]] = i;
            break;
        }
    }
    
    //利用height[rk[i+1]]>=height[rk[i]]-1
	//O(n)
	for(int i = 1, k = 0;i <= n; i++){
		if(k) k--;
		while(rk[i] > 1 && s[i + k] == s[sa[rk[i] - 1] + k]) k++;
		height[rk[i]] = k;
	}
}
\end{lstlisting}
\subsection{王老师模板}
\begin{lstlisting}
#define MAXN 1000005
//sa:排名对应的前缀, rk:前缀的排名, tp:第二关键字排名对应的前缀, tax:排名对应的个数
//height:排名i与i-1后缀的LCP(最长公共前缀)
int sa[MAXN],r1[MAXN],r2[MAXN],tax[MAXN],height[MAXN];
int *rk=r1,*tp=r2;
char s[MAXN];

void rsort(int n,int m){
	memset(tax,0,(m+1)*sizeof(tax[0]));
	for(int i=1;i<=n;i++) tax[rk[i]]++;//当前排名装桶
	for(int i=1;i<=m;i++) tax[i]+=tax[i-1];//计算桶的名次
	for(int i=n;i>=1;i--) sa[tax[rk[tp[i]]]--]=tp[i];//按照第二关键字降序,分配排名。
}
void get_sa(char* s){
	//O(nlogn)
	int n=strlen(s+1), m=0;
	for(int i=1;i<=n;i++)
		m=max(m,rk[i]=s[i]),tp[i]=i;
	rsort(n,m);
	for(int k=1,p=0;p<n;k<<=1,m=p){
		p=0;
		//重制第二关键字
		for(int i=n-k+1;i<=n;i++) tp[++p]=i; //后续为空,排前面
		for(int i=1;i<=n;i++) if(sa[i]>k) tp[++p]=sa[i]-k; //按照第一关键字排第二关键字

		rsort(n,m);
	
		swap(tp,rk);
		rk[sa[1]]=p=1;
		for(int i=2;i<=n;i++){
			rk[sa[i]]=(tp[sa[i]]==tp[sa[i-1]] && tp[sa[i]+k]==tp[sa[i-1]+k])?p:++p;
		}
	}

	//利用height[rk[i+1]]>=height[rk[i]]-1
	//O(n)
	for(int i=1,k=0;i<=n;i++){
		if(k) k--;
		while(rk[i]>1 && s[i+k]==s[sa[rk[i]-1]+k]) k++;
		height[rk[i]]=k;
	}
}
\end{lstlisting}