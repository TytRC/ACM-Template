\section{前缀函数与KMP算法}
\subsection{前缀函数求解}
\begin{lstlisting}
// s下标从0开始
for (int i = 1, j = 0; s[i]; ++i){
    while (j && s[i] != s[j]) j = pi[j - 1];
    pi[i] = (j += s[i] == s[j]);
}
\end{lstlisting}
\subsection{字符串周期求解}
$n-\pi[n-1]$是$s$的最小周期。\par
如果要求所有的周期,则有:
\begin{lstlisting}
int len = i;
while (i){
    i = pi[i - 1];
    q.push(i);
}
cout << q.size() << endl;
while (!q.empty()){
    cout << len - q.front();
    q.pop();
    if (!q.empty()) cout << " ";
}
\end{lstlisting}\par
如果要求字符串的循环节每次都完整出现,则需要$n\mod n-\pi[n-1]==0$。

\subsection{模板 By pllj}
\begin{lstlisting}
const int L = 1e6 + 7;
int nxt[L];
void getNext(char *s){
    int l = strlen(s + 1);
    int j = -1;
    nxt[0] = -1;
    for(int i = 1; i <= l; i++){
        while(~j && s[i] != s[j + 1]) j = nxt[j];
        nxt[i] = ++j;
    }
}
int kmp(char *s, char *t){
    int l = strlen(s + 1), m = strlen(t + 1);
    int res = 0; //匹配次数
    int j = 0;
    for(int i = 1; i <= l; i++){
        while(~j && s[i] != t[j + 1]) j = nxt[j];
        if(++j == m) res++;
    }
    return j;
}
\end{lstlisting}

\section{拓展KMP}
\subsection{定义}
约定:字符串下标以 $0$ 为起点。

对于个长度为 $n$ 的字符串 $s$。定义函数 $z[i]$ 表示 $s$ 和 $s[i:n-1]$(即以 $s[i]$ 开头的后缀)的最长公共前缀(LCP)的长度。$z$ 被称为 $s$ 的 \textbf{Z 函数}。特别地,$z[0] = 0$。

国外一般将计算该数组的算法称为 \textbf{Z Algorithm},而国内则称其为 \textbf{扩展 KMP}。

\subsection{O(n)求 Z 函数}
\begin{lstlisting}
const int N = 2E7 + 7;
int z[N * 2]; // 应该和题目有关
void getZ(const char* s){
    int l = 0, r = 0;
    int len = strlen(s + 1);
    z[1] = len;  // 根据题目要求决定
    for(int i = 2; i <= len; i++){
        if(i <= r) z[i] = min(z[i - l + 1], r - i + 1);
        else z[i] = 0;
        while(i + z[i] <= len && s[i + z[i]] == s[1 + z[i]]) z[i]++;
        if(i + z[i] - 1 > r){
            l = i; r = i + z[i] - 1;
        }
    }
}
\end{lstlisting}

\subsection{应用}
\begin{itemize}
	\item 匹配所有子串
	
		为了避免混淆,我们将 $t$ 称作 \textbf{文本},将 $p$ 称作 \textbf{模式}。所给出的问题是:寻找在文本 $t$ 中模式 $p$ 的所有出现。
		
		为了解决该问题,我们构造一个新的字符串 $s = p + \# + t$,也即我们将 $p$ 和 $t$ 连接在一起,但是在中间放置了一个分割字符 $\#$(我们将如此选取 $\#$ 使得其必定不出现在 $p$ 和 $t$ 中)。
		
		首先计算 $s$ 的 Z 函数。接下来,对于在区间 $[0, |t| - 1]]$ 中的任意 $i$,我们考虑以 $t[i]$ 为开头的后缀在 $s$ 中的 Z 函数值 $k=z[i + |p| + 1]$。如果 $k = |p|$,那么我们知道有一个 $p$ 的出现位于 $t$ 的第 $i$ 个位置,否则没有 $p$ 的出现位于 $t$ 的第 $i$ 个位置。
	\item 本质不同子串数
	\item 字符串整周期
\end{itemize}