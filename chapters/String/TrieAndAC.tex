\section{Trie树}
\subsection{普通模板(转自OIwiki)}
\begin{lstlisting}
struct trie {
  int nex[100000][26], cnt;
  bool exist[100000];  // 该结点结尾的字符串是否存在

  void insert(char *s, int l) {  // 插入字符串
    int p = 0;
    for (int i = 0; i < l; i++) {
      int c = s[i] - 'a';
      if (!nex[p][c]) nex[p][c] = ++cnt;  // 如果没有,就添加结点
      p = nex[p][c];
    }
    exist[p] = 1;
  }
  bool find(char *s, int l) {  // 查找字符串
    int p = 0;
    for (int i = 0; i < l; i++) {
      int c = s[i] - 'a';
      if (!nex[p][c]) return 0;
      p = nex[p][c];
    }
    return exist[p];
  }
};
\end{lstlisting}

\subsection{维护异或和}
\begin{lstlisting}
namespace xorvTrie{  // 自低位到高位依次插入,支持全局加1
    const int D = 20;
    const int SZE = N * D;
    int rt[N], ch[SZE][2], cnt[SZE], xorv[SZE];
    int tot = 0;
    void maintain(int o){
        cnt[o] = xorv[o] = 0;
        if(ch[o][0]){
            cnt[o] += cnt[ch[o][0]];
            xorv[o] ^= xorv[ch[o][0]] << 1;
        }
        if(ch[o][1]){
            cnt[o] += cnt[ch[o][1]];
            xorv[o] ^= (xorv[ch[o][1]] << 1) | (cnt[ch[o][1]] & 1);
        }
    }
    int mknode(){
        ++tot;
        ch[tot][0] = ch[tot][1] = 0; cnt[tot] = 0;
        return tot;
    }
    void insert(int &o, int x, int dep){
        if(!o) o = mknode();
        if(dep > D){
            cnt[o]++;
            return;
        }
        insert(ch[o][x & 1], x >> 1, dep + 1);
        maintain(o);
    }
    void erase(int o, int x, int dep){
        if(dep > D){
            cnt[o]--;
            return;
        }
        erase(ch[o][x & 1], x >> 1, dep + 1);
        maintain(o);
    }
    void addall(int o){
        swap(ch[o][0], ch[o][1]);
        if(ch[o][0]) addall(ch[o][0]);
        maintain(o);
    }
}
\end{lstlisting}

\section{AC自动机}
\subsection{判断有几个模式串在文本串t中出现}
\begin{lstlisting}
const int N = 1E6 + 7;
const int S = 26;
int tr[N][S], ex[N], fail[N], tot;
void insert(const char* s){
    int u = 0;
    for(int i = 0; s[i]; i++){
        int v = s[i] - 'a';
        if(!tr[u][v]) tr[u][v] = ++tot;
        u = tr[u][v];
    }
    ex[u]++;
}
queue<int> q;
void build(){
    for(int i = 0; i < S; i++) if(tr[0][i]) q.push(tr[0][i]);
    while(!q.empty()){
        int u = q.front(); q.pop();
        for(int i = 0; i < S; i++){
            if(tr[u][i]){
                fail[tr[u][i]] = tr[fail[u]][i];
                q.push(tr[u][i]);
            }
            else tr[u][i] = tr[fail[u]][i];
        }
    }
}
int query(const char* t){
    int res = 0;
    for(int i = 0, u = 0; t[i]; i++){
        int v = t[i] - 'a';
        u = tr[u][v];
        for(int j = u; j && ~ex[j]; j = fail[j]){
            res += ex[j]; ex[j] = -1;
        }
    }
    return res;
}
\end{lstlisting}

\subsection{判断模式串在文本串出现次数}
\begin{lstlisting}
const int N = 1005;
const int S = 26;
const int L = 55;
const int W = N * L;
const int M = 2e6 + 7;

int tr[W][S], tot, fail[W], pos[N];
void insert(const char* s, int x){
    int u = 0;
    for(int i = 0; s[i]; i++){
        int v = s[i] - 'A';
        if(!tr[u][v]) tr[u][v] = ++tot;
        u = tr[u][v];
    }
    pos[x] = u;
}
queue<int> q;
vector<vector<int> > e;
void build(){
    for(int i = 0; i < S; i++) if(tr[0][i]) q.push(tr[0][i]);
    while(!q.empty()){
        int u = q.front(); q.pop();
        e[fail[u]].push_back(u);
        for(int i = 0; i < S; i++){
            if(tr[u][i]){
                fail[tr[u][i]] = tr[fail[u]][i];
                q.push(tr[u][i]);
            }
            else tr[u][i] = tr[fail[u]][i];
        }
    }
}
int cnt[W];
void query(const char* t){
    int u = 0;
    for(int i = 0; t[i]; i++){
        if(t[i] < 'A' || t[i] > 'Z'){
            u = 0; continue;  // 题目保证模式串中只含大写字母
        }
        u = tr[u][t[i] - 'A'];
        cnt[u]++;
    }
}
void dfs(int x){
    for(auto v: e[x]){
        dfs(v);
        cnt[x] += cnt[v];
    }
}
\end{lstlisting}

\subsection{模板 By RainCurtain}
\begin{lstlisting}
struct AC{
    int tr[MAXN][26], cnt;
    int id[MAXN], fail[MAXN];
    int val[MAXN], num[155]; // val表示这个节点被匹配了几次,num表示这个模式串被匹配了几次

    inline void init(){ /**code here**/ }

    inline void insert(const char* s, int poi){
        int p = 0;
        for (int i = 0; s[i]; ++i){
            int c = s[i] - 'a';
            if (!tr[p][c]) tr[p][c] = ++cnt;
            p = tr[p][c];
        }
        id[p] = poi;
    }
    inline void build(){
        queue<int> q;
        for (int i = 0; i < 26; ++i)
            if (tr[0][i]) q.push(tr[0][i]);
        while (!q.empty()){
            int u = q.front();
            q.pop();
            for (int i = 0; i < 26; ++i){
                int v = tr[u][i];
                if (v){
                    fail[v] = tr[fail[u]][i];
                    q.push(v);
                }else tr[u][i] = tr[fail[u]][i];
            }
        }
    }
    int query(const char* s){
        int p = 0;
        for (int i = 0; s[i]; ++i){
            int c = s[i] - 'a';
            p = tr[p][c];
            for (int j = p; j; j = fail[j]) ++val[j];
        }
        int res = 0;
        for (int i = 0; i <= cnt; ++i) {
            if (id[i]) {
                res = max(res, val[i]);
                num[id[i]] = val[i];
            }
        }
        return  res;
    }
}
\end{lstlisting}