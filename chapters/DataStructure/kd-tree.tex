\section{KD-Tree}
\subsection{K远点对}
已知平面内 $N$ 个点的坐标,求欧氏距离下的第 $K$ 远点对。
\begin{lstlisting}
#include<iostream>
#include<cstdio>
#include<algorithm>
#include<queue>
using namespace std;
int ndm,cnt,root;
struct Point{
	long long x[2];
}p[100005];
bool operator<(const Point &a,const Point &b){
	return a.x[ndm]<b.x[ndm];
}
priority_queue<long long,vector<long long>,greater<long long> > q;
struct Node{
	long long mx[2],mn[2];
	int ls,rs,sz;
	Point p;
}t[100005];
bool cmp(const Point &a,const Point &b){
	return a.x<b.x;
}
long long sqr(long long x){
	return x*x;
}
long long abss(long long x){
	return x<0?-x:x;
}
long long dis(Point a,Point b){
	long long res=0;
	for(int i=0;i<2;i++) res+=sqr(a.x[i]-b.x[i]);
	return res;
}
long long mxdis(int o,int y){
	long long res=0;
	for(int i=0;i<2;i++){
		res+=sqr(max(abss(t[o].mx[i]-p[y].x[i]),abss(t[o].mn[i]-p[y].x[i])));
	}
	return res;
}
void pushup(int x){
	int ls=t[x].ls,rs=t[x].rs;
	for(int i=0;i<2;i++){
		t[x].mn[i]=t[x].mx[i]=t[x].p.x[i];
		if(ls){
			t[x].mn[i]=min(t[x].mn[i],t[ls].mn[i]);
			t[x].mx[i]=max(t[x].mx[i],t[ls].mx[i]);
		}
		if(rs){
			t[x].mn[i]=min(t[x].mn[i],t[rs].mn[i]);
			t[x].mx[i]=max(t[x].mx[i],t[rs].mx[i]);
		}
	}
	t[x].sz=t[ls].sz+t[rs].sz+1;
}
int build(int l,int r,int dim){
	if(l>r) return 0;
	int mid=(l+r)>>1;
	int rt=++cnt;
	ndm=dim;
	nth_element(p+l,p+mid,p+r+1);
	t[rt].p=p[mid];
	t[rt].ls=build(l,mid-1,dim^1);
	t[rt].rs=build(mid+1,r,dim^1);
	pushup(rt);
	return rt;
}
void solve(int o,int v){
	long long dl=-1,dr=-1;
	if(t[o].ls) dl=mxdis(t[o].ls,v);
	if(t[o].rs) dr=mxdis(t[o].rs,v);
	long long dn=dis(t[o].p,p[v]);
	if(dn>q.top()){
		q.pop();
		q.push(dn);
	}
	if(dl>dr){
		if(dl>q.top()) solve(t[o].ls,v);
		if(dr>q.top()) solve(t[o].rs,v);
	}
	else{
		if(dr>q.top()) solve(t[o].rs,v);
		if(dl>q.top()) solve(t[o].ls,v);
	}
	
}
int main(){
	int i,j,k,m,n;
	scanf("%d%d",&n,&k);
	for(i=1;i<=n;i++){
		scanf("%lld%lld",&p[i].x[0],&p[i].x[1]);
	}
	//random_shuffle(p+1,p+n+1);
	root=build(1,n,0);
	for(i=1;i<=2*k;i++) q.push(0);
	for(i=1;i<=n;i++){
		solve(root,i);
	}
	printf("%lld\n",q.top());
	return 0;
}
\end{lstlisting}