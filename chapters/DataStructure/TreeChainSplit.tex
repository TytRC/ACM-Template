\section{树链剖分}
\subsection{剖分一}
已知一棵包含 $N$ 个结点的树(连通且无环),每个节点上包含一个数值,需要支持以下操作:
\begin{enumerate}[操作1.]
    \item \textit{1 x y z} 表示将树从 $x$ 到 $y$ 结点最短路径上所有节点的值都加上 $z$。
    \item \textit{2 x y} 表示求树从 $x$ 到 $y$ 结点最短路径上所有节点的值之和。
    \item \textit{3 x z} 表示将以 $x$ 为根节点的子树内所有节点值都加上 $z$。
    \item \textit{4 x} 表示求以 $x$ 为根节点的子树内所有节点值之和。
\end{enumerate}

\begin{lstlisting}
    #include<iostream>
    #include<cstdio>
    #define ls (o<<1)
    #define rs ((o<<1)+1)
    using namespace std;
    const int N=1e5+7;
    long long a[N];
    long long p;
    int n;
    struct TREE{
        int nxt,to;
    }t[N<<1];
    int last[N],cnt;
    void build(int u,int v){
        t[++cnt].to=v;
        t[cnt].nxt=last[u];
        last[u]=cnt;
    }
    int f[N],sze[N],dep[N],son[N],id[N],idx=0,top[N];long long na[N];
    void dfs1(int fa,int u){
        f[u]=fa;dep[u]=dep[fa]+1;sze[u]=1;
        int ms=0;
        int e=last[u];
        while(e){
            int v=t[e].to;
            if(v!=fa){
                dfs1(u,v);
                sze[u]+=sze[v];
                if(sze[v]>ms){
                    son[u]=v;
                    ms=sze[v];
                }
            }
            e=t[e].nxt;
        }
    }
    void dfs2(int x,int topf){
        id[x]=++idx;
        na[idx]=a[x];
        top[x]=topf;
        if(son[x]!=0) dfs2(son[x],topf);
        else return;
        int e=last[x];
        while(e){
            int v=t[e].to;
            if(v!=son[x]&&v!=f[x]){
                dfs2(v,v);
            }
            e=t[e].nxt;
        }
    }
    long long tr[N<<2],tag[N<<2];
    void pushup(int o){
        tr[o]=(tr[ls]+tr[rs])%p;
    }
    void buildt(int o,int l,int r){
        if(l==r){
            tr[o]=na[l];
            tag[o]=0;
            return;
        }
        int mid=(l+r)>>1;
        buildt(ls,l,mid);
        buildt(rs,mid+1,r);
        pushup(o);
    }
    void pushdown(int o,int l,int r){
        int mid=(l+r)>>1;
        tr[ls]=(tr[ls]+tag[o]*(mid-l+1))%p;
        tr[rs]=(tr[rs]+tag[o]*(r-mid))%p;
        tag[ls]=(tag[ls]+tag[o])%p;
        tag[rs]=(tag[rs]+tag[o])%p;
        tag[o]=0;
    }
    void update(int o,int l,int r,int cl,int cr,long long w){
        if(l>cr||r<cl) return;
        if(cl<=l&&r<=cr){
            tag[o]=(tag[o]+w)%p;
            tr[o]=(tr[o]+(r-l+1)*w)%p;
            return;
        }
        pushdown(o,l,r);
        int mid=(l+r)>>1;
        if(cl<=mid) update(ls,l,mid,cl,cr,w);
        if(cr>mid) update(rs,mid+1,r,cl,cr,w);
        pushup(o);
    }
    long long query(int o,int l,int r,int ql,int qr){
        if(l>qr||r<ql) return 0;
        if(ql<=l&&r<=qr) return tr[o];
        pushdown(o,l,r);
        int mid=(l+r)>>1;
        long long res=0;
        if(ql<=mid) res+=query(ls,l,mid,ql,qr);
        if(qr>mid) res+=query(rs,mid+1,r,ql,qr);
        return res%p;
    }
    void updRange(int x,int y,long long k){
        while(top[x]!=top[y]){
            if(dep[top[x]]<dep[top[y]]) swap(x,y);
            update(1,1,n,id[top[x]],id[x],k);
            x=f[top[x]];
        }
        if(dep[x]>dep[y]) swap(x,y);
        update(1,1,n,id[x],id[y],k);
    }
    long long qRange(int x,int y){
        long long ans=0;
        while(top[x]!=top[y]){
            if(dep[top[x]]<dep[top[y]]) swap(x,y);
            ans=(ans+query(1,1,n,id[top[x]],id[x]))%p;
            x=f[top[x]];
        }
        if(dep[x]>dep[y]) swap(x,y);
        ans=(ans+query(1,1,n,id[x],id[y]))%p;
        return ans;
    }
    void updSon(int x,long long k){
        update(1,1,n,id[x],id[x]+sze[x]-1,k);
    }
    long long qSon(int x){
        return query(1,1,n,id[x],id[x]+sze[x]-1);
    }
    int main(){
        int i,j,k,m,r,x,y,opt;long long z;
        scanf("%d%d%d%d",&n,&m,&r,&p);
        for(i=1;i<=n;i++) scanf("%lld",&a[i]);
        for(i=1;i<n;i++){
            scanf("%d%d",&x,&y);
            build(x,y);build(y,x);
        }
        dfs1(0,r);
        dfs2(r,r);
        buildt(1,1,n);
        for(i=1;i<=m;i++){
            scanf("%d",&opt);
            if(opt==1){
                scanf("%d%d%lld",&x,&y,&z);
                updRange(x,y,z);
            }
            else if(opt==2){
                scanf("%d%d",&x,&y);
                printf("%lld\n",qRange(x,y));
            }
            else if(opt==3){
                scanf("%d%lld",&x,&z);
                updSon(x,z);
            }
            else if(opt==4){
                scanf("%d",&x);
                printf("%lld\n",qSon(x));
            }
        }
        return 0;
    }
\end{lstlisting}

\subsection{剖分二}
有一棵点数为 $N$ 的树,以点 $1$ 为根,且树点有边权。\\
然后有 $M$ 个 操作,分为三种: 
\begin{enumerate}[操作1.]
    \item 把某个节点 $x$ 的点权增加 $a$
    \item 把某个节点 $x$ 为根的子树中所有点的点权都增加 $a$ 
    \item 询问某个节点 $x$ 到根的路径中所有点的点权和
\end{enumerate}
\begin{lstlisting}
#include <iostream>
#include <cstdio>
#include <vector>
#define ls (o << 1)
#define rs ((o << 1) | 1)
using namespace std;
const int N = 1e5 + 7;
vector<vector<int> > e;
long long w[N];
int sze[N], bc[N], dep[N], id[N], rv[N], top[N], pa[N], mk;
struct SGT{
    long long sum, tag;
}t[N << 2];
void pre(int x, int fa){
//    cout << x << " " << fa << endl;
    dep[x] = dep[fa] + 1;
    sze[x] = 1;
    pa[x] = fa;
    for(auto v: e[x]){
        if(v == fa) continue;
        pre(v, x);
        if(sze[v] > sze[bc[x]]) bc[x] = v;
        sze[x] += sze[v];
    }
}
void dfs(int x, int fa, int tp){
//    cout << x << " " << fa << " " << tp << endl;
    id[x] = ++mk;
    rv[mk] = x;
    top[x] = tp;
    if(bc[x]) dfs(bc[x], x, tp);
    for(auto v: e[x]){
        if(v == fa || v == bc[x]) continue;
        dfs(v, x, v);
    }
}
void pushup(int o){
    t[o].sum = t[ls].sum + t[rs].sum;
}
void pushdown(int o, int l, int r){
    if(t[o].tag == 0) return;
    int mid = (l + r) >> 1;
    t[ls].sum += t[o].tag * (mid - l + 1);
    t[rs].sum += t[o].tag * (r - mid);
    t[ls].tag += t[o].tag;
    t[rs].tag += t[o].tag;
    t[o].tag = 0;
}
void build(int o, int l, int r){
    if(l == r){
        t[o].sum = w[rv[l]];
        t[o].tag = 0;
        return;
    }
    int mid = (l + r) >> 1;
    build(ls, l, mid); build(rs, mid + 1, r);
    pushup(o);
}
void update(int o, int l, int r, int ql, int qr, long long val){
    if(ql > r || qr < l) return;
    if(ql <= l && r <= qr){
        t[o].sum += val * (r - l + 1);
        t[o].tag += val;
        return;
    }
    pushdown(o, l, r);
    int mid = (l + r) >> 1;
    if(ql <= mid) update(ls, l, mid, ql, qr, val);
    if(qr > mid) update(rs, mid + 1, r, ql, qr, val);
    pushup(o);
}
long long query(int o, int l, int r, int ql, int qr){
    if(ql > r || qr < l) return 0;
    if(ql <= l && r <= qr) return t[o].sum;
    pushdown(o, l, r);
    int mid = (l + r) >> 1;
    long long res = 0;
    if(ql <= mid) res += query(ls, l, mid, ql, qr);
    if(qr > mid) res += query(rs, mid + 1, r, ql, qr);
    pushup(o);
    return res;
}
int main(){
    int n, m, u, v, opt; long long val;
    scanf("%d%d", &n, &m);
    e.resize(n + 1);
    for(int i = 1; i <= n; i++) scanf("%lld", &w[i]);
    for(int i = 1; i < n; i++){
        scanf("%d%d", &u, &v);
        e[u].emplace_back(v);
        e[v].emplace_back(u);
    }
    pre(1, 0);
    dfs(1, 0, 1);
    build(1, 1, n);
//    for(int i = 1; i <= n; i++) cout << i << " " << top[i] << endl;
    for(int i = 1; i <= m; i++){
        scanf("%d", &opt);
        if(opt == 1){
            scanf("%d%lld", &u, &val);
            update(1, 1, n, id[u], id[u], val);
        }
        else if(opt == 2){
            scanf("%d%lld", &u, &val);
            update(1, 1, n, id[u], id[u] + sze[u] - 1, val);
        }
        else if(opt == 3){
            scanf("%d", &u);
            long long ans = 0;
            while(top[u] != 1){
                ans += query(1, 1, n, id[top[u]], id[u]);
                u = pa[top[u]];
//                cout << u << endl;
            }
            ans += query(1, 1, n, id[top[u]], id[u]);
            printf("%lld\n", ans);
        }
    }
    return 0;
}
\end{lstlisting}