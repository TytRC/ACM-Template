\section{莫队算法}
\subsection{普通莫队}
	\href{https://darkbzoj.tk/problem/2038}{黑暗爆炸 - 2038 小Z的袜子} \par
	查询区间中任意取两数相等的概率
	\begin{lstlisting}
#include <iostream>
#include <cstdio>
#include <cmath>
#include <algorithm>
using namespace std;
const int N = 1E5 + 7;
int c[N], pos[N];
struct QRY{
    int l, r, id;
    bool operator < (const QRY &tmp)const{
        return pos[l] == pos[tmp.l]?r < tmp.r:l < tmp.l;
    }
}q[N];
struct ANS{
    long long a, b;
}ans[N];
long long gcd(long long a, long long b){
    return b == 0?a:gcd(b, a % b);
}
long long cnt[N];
long long sum = 0;
void add(int x){
    sum += cnt[x];
    cnt[x]++;
}
void del(int x){
    cnt[x]--;
    sum -= cnt[x];
}
int main(){
    int n, m, l, r;
    scanf("%d%d", &n, &m);
    int t = sqrt(n);
    for(int i = 1; i <= n; i++){
        scanf("%d", &c[i]);
        pos[i] = i / t + 1;
    }
    for(int i = 1; i <= m; i++){
        scanf("%d%d", &l, &r);
        ans[i].b = 1;
        q[i] = {l, r, i};
    }
    sort(q + 1, q + m + 1);
    l = 1, r = 0;
    for(int i = 1; i <= m; i++){
        if(q[i].l == q[i].r) continue;
        while(l > q[i].l) add(c[--l]);
        while(r < q[i].r) add(c[++r]);
        while(l < q[i].l) del(c[l++]);
        while(r > q[i].r) del(c[r--]);
        ans[q[i].id] = {sum, (long long)(q[i].r - q[i].l + 1) * (q[i].r - q[i].l) / 2};
        long long g = gcd(ans[q[i].id].a, ans[q[i].id].b);
        ans[q[i].id].a /= g; ans[q[i].id].b /= g;
    }
    for(int i = 1; i <= m; i++) printf("%lld/%lld\n", ans[i].a, ans[i].b);
    return 0;
}\end{lstlisting}
\subsection{莫队+数据结构}
	\href{https://darkbzoj.tk/problem/3809}{BZOJ3809 Gty的二逼妹子序列} \par
	查询区间内值在 $[a, b]$ 范围内的种类数,可用莫队+树状数组维护
	\begin{lstlisting}
#include <iostream>
#include <cstdio>
#include <cmath>
#include <algorithm>
using namespace std;
const int N = 3E5 + 7;
int s[N], pos[N];
struct QRY{
    int l, r, a, b, id;
    bool operator < (const QRY &tmp) const{
        return pos[l] == pos[tmp.l]?r < tmp.r:l < tmp.l;
    }
}q[N * 4];
int hv[N], block[N];
int ans[N * 4];
void add(int x){
    if(!hv[x]) block[pos[x]]++;
    hv[x]++;
}
void del(int x){
    hv[x]--;
    if(!hv[x]) block[pos[x]]--;
}
int main(){
    int n, m, l, r, a, b;
    scanf("%d%d", &n, &m);
    int t = sqrt(n);
    for(int i = 1; i <= n; i++){
        scanf("%d", &s[i]);
        pos[i] = i / t + 1;
    }
    for(int i = 1; i <= m; i++){
        scanf("%d%d%d%d", &l, &r, &a, &b);
        q[i] = {l, r, a, b, i};
    }
    sort(q + 1, q + m + 1);
    l = 0, r = 0;
    for(int i = 1; i <= m; i++){
        while(l > q[i].l) add(s[--l]);
        while(r < q[i].r) add(s[++r]);
        while(l < q[i].l) del(s[l++]);
        while(r > q[i].r) del(s[r--]);
        a = q[i].a; b = q[i].b;
        int res = 0;
        if(pos[a] == pos[b]){
            while(a <= b){
                if(hv[a]) res++;
                a++;
            }
            ans[q[i].id] = res;
            continue;
        }
        int x = pos[a] * t, y = (pos[b] - 1) * t;
        while(a < x && a <= n){
            if(hv[a]) res++;
            a++;
        }
        while(y <= b){
            if(hv[y]) res++;
            y++;
        }
        for(int j = pos[q[i].a] + 1; j < pos[q[i].b]; j++){
            res += block[j];
        }
        ans[q[i].id] = res;
    }
    for(int i = 1; i <= m; i++){
        printf("%d\n", ans[i]);
    }
}\end{lstlisting}
\subsection{带修莫队}
	\href{https://codeforces.com/problemset/problem/940/F}{CodeForces - 940F Machine Learning} \par
	查询区间数出现次数的MEX(因答案不超过1000,故可以每次暴力查询)
	\begin{lstlisting}
#include <iostream>
#include <cstdio>
#include <algorithm>
#include <cmath>
#include <unordered_map>
using namespace std;
const int N = 2e5 + 7;
int pos[N], a[N];
int chp[N], chx[N];
int hv[N], ct[N];
struct QRY{
    int l, r, t, id;
    bool operator < (const QRY &tmp)const{
        if(pos[l] != pos[tmp.l]) return l < tmp.l;
        if(pos[r] != pos[tmp.r]) return r < tmp.r;
        return t < tmp.t;
    }
}q[N];
int cnt, tt, tot, res = 1, l, r;
unordered_map<int, int> ma;
int ans[N];
void add(int x){
    int tmp = ct[x]++;
    hv[tmp]--;
    hv[tmp + 1]++;
}
void del(int x){
    int tmp = ct[x]--;
    hv[tmp]--;
    hv[tmp - 1]++;
}
void work(int x){
    if(l <= chp[x] && chp[x] <= r){
        del(a[chp[x]]);
        add(chx[x]);
    }
    swap(a[chp[x]], chx[x]);
}
int main(){
    int n, m, t = 0, opt, rr;
    scanf("%d%d", &n, &m);
    int block_size = pow(n, 2.0 / 3);
    for(int i = 1; i <= n; i++){
        scanf("%d", &rr);
        if(ma[rr]) a[i] = ma[rr];
        else{
            a[i] = ++tot;
            ma[rr] = tot;
        }
        pos[i] = i / block_size + 1;
    }
    for(int i = 1; i <= m; i++){
        scanf("%d%d%d", &opt, &l, &r);
        if(opt == 1){
            cnt++;
            q[cnt] = {l, r, tt, cnt};
        }
        else{
            tt++;
            chp[tt] = l;
            if(ma[r]) chx[tt] = ma[r];
            else{
                ma[r] = ++tot;
                chx[tt] = tot;
            }
        }
    }
    sort(q + 1, q + cnt + 1);
    l = r = t = 0;
    hv[0] = 1e9;
    for(int i = 1; i <= cnt; i++){
        while(l > q[i].l) add(a[--l]);
        while(r < q[i].r) add(a[++r]);
        while(l < q[i].l) del(a[l++]);
        while(r > q[i].r) del(a[r--]);
        while(t < q[i].t) work(++t);
        while(t > q[i].t) work(t--);
        res = 1;
        while(hv[res]) res++;
        ans[q[i].id] = res;
    }
    for(int i = 1; i <= cnt; i++) printf("%d\n", ans[i]);
}\end{lstlisting}

\subsection{回滚莫队}
\subsubsection{只使用增加操作}
	\href{https://www.luogu.com.cn/problem/AT1219}{AT1219 歴史の研究} \par
	求区间 $\max\{v * cnt_v\}$​。
	\begin{lstlisting}
#include <iostream>
#include <cstdio>
#include <cmath>
#include <algorithm>
using namespace std;
const int N = 2E5 + 7;
long long a[N], b[N];
int pos[N], block;
struct QRY{
    int l, r, id;
    bool operator<(const QRY &_)const{
        if(pos[l] != pos[_.l]) return l < _.l;
        return r < _.r;
    }
}qry[N];
long long ans[N];
int cnt[N], tcnt[N];
void add(int x, long long &v){
    cnt[a[x]]++;
    v = max(v, cnt[a[x]] * b[a[x]]);
}
void del(int x){
    cnt[a[x]]--;
}
int main(){
    int n, q, l, r;
    scanf("%d%d", &n, &q);
    block = sqrt(n);
    for(int i = 1; i <= n; i++){
        scanf("%lld", &a[i]);
        b[i] = a[i];
        pos[i] = (i - 1) / block + 1;
    }
    sort(b + 1, b + n + 1);
    int nk = unique(b + 1, b + n + 1) - b - 1;
    for(int i = 1; i <= n; i++) a[i] = lower_bound(b + 1, b + nk + 1, a[i]) - b;
    for(int i = 1; i <= q; i++){
        scanf("%d%d", &l , &r);
        qry[i] = {l, r, i};
    }
    sort(qry + 1, qry + q + 1);
    long long res = 0;
    l = 1; r = 0; int nb = 0;
    for(int i = 1; i <= q; i++){
        if(pos[qry[i].l] == pos[qry[i].r]){
            long long tmp = 0;
            for(int j = qry[i].l; j <= qry[i].r; j++){
                tcnt[a[j]]++;
                tmp = max(tmp, tcnt[a[j]] * b[a[j]]);
            }
            ans[qry[i].id] = tmp;
            for(int j = qry[i].l; j <= qry[i].r; j++) tcnt[a[j]]--;
            continue;
        }
        if(pos[qry[i].l] != nb){
            int tl = min(n + 1, pos[qry[i].l] * block + 1);
            int tr = l - 1;
            while(l < tl) del(l++);
            while(r > tr) del(r--);
            res = 0;
            nb = pos[qry[i].l];
        }
        while(r < qry[i].r) add(++r, res);
        int ll = l; long long tmp = res;
        while(ll > qry[i].l) add(--ll, tmp);
        ans[qry[i].id] = tmp;
        while(ll < l) del(ll++);
    }
    for(int i = 1; i <= q; i++) printf("%lld\n", ans[i]);
    return 0;
}\end{lstlisting}

\subsubsection{用栈记录历史修改的写法}
	给一个序列 $A$,$q$ 次询问,每次询问给一个区间 $[L,R]$, 再给一个 $k$,要求: \par
	

$$
\begin{aligned}
& \sum_{i = 0}^{k - 1}{f(\{A_{l-i}\dots A_{r + i}\}) \times (n + i)^i (\mod P)} \\
& f(S)=\max\{len|\exists x,\forall u \in \{x,x + 1,\dots,x + len - 1\}, u \in S\}
\end{aligned}
$$

	\begin{lstlisting}
#include <iostream>
#include <cstdio>
#include <cmath>
#include <algorithm>
#include <stack>
using namespace std;
const long long mod = 998244353;
const int N = 1E5 + 7;
int block;
int a[N], cnt[N], pos[N];
struct QRY{
    int l, r, k, id;
    bool operator < (const QRY &_)const{
        if(pos[l] != pos[_.l]) return l < _.l;
        return r < _.r;
    }
}q[N];
int head[N], tail[N];
long long pw[N];
stack<pair<int* , int>> st;
void clear(int sz){
    while(st.size() > sz){
        *st.top().first = st.top().second;
        st.pop();
    }
}
void upd(int* v, int x){
    st.push({v, *v});
    *v = x;
}
int res;
void add(int x){
    x = a[x];
    if(cnt[x]) return;
    upd(&cnt[x], 1);
    int r = cnt[x + 1] ? tail[x + 1] : x, l = cnt[x - 1] ? head[x - 1] : x;
    upd(&head[r], l); upd(&tail[l], r);
    if(r - l + 1 > res) upd(&res, r - l + 1);
}
long long ans[N];
int main(){
    int n, qq;
    scanf("%d%d", &n, &qq);
    block = sqrt(n);
    for(int i = 1; i <= n; i++){
        scanf("%d", &a[i]);
        pos[i] = (i - 1) / block + 1;
    }
    pw[0] = 1;
    for(int i = 1; i <= n; i++) pw[i] = pw[i - 1] * (n + 1) % mod;
    int l, r, k;
    for(int i = 1; i <= qq; i++){
        scanf("%d%d%d", &l, &r, &k);
        q[i] = {l, r, k, i};
    }
    sort(q + 1, q + qq + 1);
    l = r = 0; int lb = 0;
    for(auto sq: q){
        if(pos[sq.l] != lb){
            r = min(pos[sq.l] * block, n);
            l = r + 1;
            lb = pos[sq.l];
            clear(0);
        }
        while(r < sq.r) add(++r);
        int ts = st.size();
        for(int j = min(sq.r, l - 1); j >= sq.l; j--) add(j);
        for(int j = 0; j < sq.k; j++){
            if(j) add(sq.l - j), add(sq.r + j);
            ans[sq.id] = (ans[sq.id] + res * pw[j]) % mod;
        }
        clear(ts);
    }
    for(int i = 1; i <= qq; i++) printf("%lld\n", ans[i]);
    return 0;
}\end{lstlisting}

\section{树上莫队}
	\href{https://www.luogu.com.cn/problem/P4074}{Luogu-P4074 糖果公园} \par
	给一棵$n$​​个节点的树,每个点有一个值$C_i$​​,每次询问一条路径 $ x \rightarrow y$​​​,求$\sum_cval_c \times \sum_{i = 1}^{cnt_c}{worth_i}(cnt_c = \sum_{i \in (x \rightarrow y)}{[C_i ==c]})$​。带修改。
	
\subsection{树上问题转括号序列}
	\begin{lstlisting}
#include <iostream>
#include <cstdio>
#include <vector>
#include <cmath>
#include <algorithm>
using namespace std;
const int N = 1e5 + 7;
int vc[N], w[N];
int id[N * 2], st[N], en[N],cnt;
int f[N][22], dep[N], lg[N];
int pos[N * 2];
long long ans[N];
int vis[N];
int c[N];
struct CHG{
    int idx, val;
};
vector<vector<int> > e;
vector<CHG> ver;
void dfs(int x, int fa){
    id[st[x] = ++cnt] = x;
    dep[x] = dep[fa] + 1;
    f[x][0] = fa;
    for(int i = 1; i <= 20; i++) f[x][i] = f[f[x][i - 1]][i - 1];
    for(auto v: e[x]){
        if(v == fa) continue;
        dfs(v, x);
    }
    id[en[x] = ++cnt] = x;
}
int lca(int x, int y){
    if(dep[x] < dep[y]) swap(x, y);
    while(dep[x] > dep[y]){
        x = f[x][lg[dep[x] - dep[y]]];
    }
    if(x == y) return x;
    for(int i = 20; i >= 0; i--){
        if(f[x][i] != f[y][i]){
            x = f[x][i]; y = f[y][i];
        }
    }
    return f[x][0];
}
struct QRY{
    int l, r, t, id;
    bool operator<(const QRY &_)const{
        if(pos[l] != pos[_.l]) return l < _.l;
        if(pos[r] != pos[_.r]) return r < _.r;
        return t < _.t;
    }
};
vector<QRY> qry;
int tv = 0;
long long res = 0;
int ct[N];
void add(int x){
    if(vis[x]) res -= (long long)vc[c[x]] * w[ct[c[x]]--];
    else res += (long long)vc[c[x]] * w[++ct[c[x]]];
    vis[x] ^= 1;
}
void work(int x){
    if(vis[ver[x].idx]){
        add(ver[x].idx);
        swap(c[ver[x].idx], ver[x].val);
        add(ver[x].idx);
    }
    else swap(c[ver[x].idx], ver[x].val);
}
int main(){
    int n, m, q, x, y, opt;
    scanf("%d%d%d", &n, &m, &q);
    for(int i = 2; i <= n; i++) lg[i] = lg[i >> 1] + 1;
    e.resize(n + 1);
    for(int i = 1; i <= m; i++) scanf("%d", &vc[i]);
    for(int i = 1; i <= n; i++) scanf("%d", &w[i]);
    for(int i = 1; i < n; i++){
        scanf("%d%d", &x, &y);
        e[x].push_back(y);
        e[y].push_back(x);
    }
    for(int i = 1; i <= n; i++) scanf("%d", &c[i]);
    dfs(1, 0);
    int blk = pow(cnt, 2.0 / 3);
    for(int i = 1; i <= cnt; i++) pos[i] = i / blk + 1;
    ver.push_back({0, 0});
    for(int i = 1; i <= q; i++){
        scanf("%d%d%d", &opt, &x, &y);
        if(opt == 0){
            tv++;
            ver.push_back({x, y});
        }
        else{
            if(st[x] > st[y]) swap(x, y);
            qry.push_back({lca(x, y) == x?st[x]:en[x], st[y], tv, (int)qry.size()});
        }
    }
    sort(qry.begin(), qry.end());
    int l, r, t;
    l = r = 0; t = 0;
    for(auto sq: qry){
        while(t < sq.t) work(++t);
        while(t > sq.t) work(t--);
        while(l > sq.l) add(id[--l]);
        while(r < sq.r) add(id[++r]);
        while(l < sq.l) add(id[l++]);
        while(r > sq.r) add(id[r--]);
        x = id[sq.l], y = id[sq.r];
        int llca = lca(x, y);
        if(x != llca && y != llca){
            add(llca);
            ans[sq.id] = res;
            add(llca);
        }
        else ans[sq.id] = res;
    }
    for(int i = 0; i < qry.size(); i++) printf("%lld\n", ans[i]);
}	
\end{lstlisting}

\subsection{树上分块}
	\begin{lstlisting}
#include <iostream>
#include <cstdio>
#include <vector>
#include <stack>
#include <cmath>
#include <algorithm>
using namespace std;
const int N = 1E5 + 7;
int val[N], w[N], c[N];
long long ans[N];
long long res;
vector<vector<int> > e;
stack<int> st;
int block_size, blk;
int dep[N], sze[N], bc[N], bl[N], f[N];
void dfs1(int x, int fa){
    dep[x] = dep[fa] + 1;
    f[x] = fa;
    sze[x] = 1;
    int ss = st.size();
    for(auto v: e[x]){
        if(v == fa) continue;
        dfs1(v, x);
        if(sze[bc[x]] < sze[v]) bc[x] = v;
        sze[x] += sze[v];
        if(st.size() >= ss + block_size){
            blk++;
            while(st.size() != ss){
                bl[st.top()] = blk;
                st.pop();
            }
        }
    }
    st.push(x);
}
int id[N], top[N], cnt;
void dfs2(int x, int rt){
    id[x] = ++cnt;
    top[x] = rt;
    if(bc[x]) dfs2(bc[x], rt);
    for(auto v: e[x]){
        if(v == f[x] || v == bc[x]) continue;
        dfs2(v, v);
    }
}
int lca(int x, int y){
    while(top[x] != top[y]){
        if(dep[top[x]] < dep[top[y]]) swap(x, y);
        x = f[top[x]];
    }
    return dep[x] < dep[y] ? x : y;
}
struct VT{
    int x, y;
};
vector<VT> ve;
struct QRY{
    int x, y, t, id;
    bool operator<(const QRY &r)const{
        if(bl[x] != bl[r.x]) return bl[x] < bl[r.x];
        if(bl[y] != bl[r.y]) return bl[y] < bl[r.y];
        return t < r.t;
    }
};
vector<QRY> qry;
bool vis[N];
int ct[N];
void update(int x){
    if(vis[x]) res -= (long long)w[ct[c[x]]--] * val[c[x]];
    else res += (long long)w[++ct[c[x]]] * val[c[x]];
    vis[x] ^= 1;
}
void move(int x, int y){
    if(dep[x] < dep[y]) swap(x, y);
    while(dep[x] > dep[y]){
        update(x);
        x = f[x];
    }
    while(x != y){
        update(x); update(y);
        x = f[x]; y = f[y];
    }
}
void make(int t){
    if(vis[ve[t].x]){
        update(ve[t].x);
        swap(ve[t].y, c[ve[t].x]);
        update(ve[t].x);
    }
    else swap(ve[t].y, c[ve[t].x]);
}
int main(){
    int n, m, q, u, v;
    scanf("%d%d%d", &n, &m, &q);
    e.resize(n + 1);
    block_size = (int)pow(n, 0.6);
    for(int i = 1; i <= m; i++) scanf("%d", &val[i]);
    for(int i = 1; i <= n; i++) scanf("%d", &w[i]);
    for(int i = 1; i < n; i++){
        scanf("%d%d", &u, &v);
        e[u].push_back(v); e[v].push_back(u);
    }
    for(int i = 1; i <= n; i++) scanf("%d", &c[i]);
    dfs1(1, 1);
    while(!st.empty()){
        blk++;
        bl[st.top()] = blk;
        st.pop();
    }
    dfs2(1, 1);
    int opt, x, y;
    int ver = 0;
    ve.push_back({0, 0});
    for(int i = 1; i <= q; i++){
        scanf("%d%d%d", &opt, &x, &y);
        if(opt == 0){
            ver++; ve.push_back({x, y});
        }
        else{
            if(id[x] > id[y]) swap(x , y);
            qry.push_back({x, y, ver, (int)qry.size()});
        }
    }
    sort(qry.begin(), qry.end());
    x = 1, y = 1; int t = 0;
    for(int i = 0; i < qry.size(); i++){
        if(x != qry[i].x) move(x, qry[i].x), x = qry[i].x;
        if(y != qry[i].y) move(y, qry[i].y), y = qry[i].y;
        int llca = lca(x, y);
        update(llca);
        while(t < qry[i].t) make(++t);
        while(t > qry[i].t) make(t--);
        ans[qry[i].id] = res;
        update(llca);
    }
    for(int i = 0; i < qry.size(); i++) printf("%lld\n", ans[i]);
    return 0;
}\end{lstlisting}