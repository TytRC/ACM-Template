\section{可持久化数据结构}
\subsection{可持久化线段树}
\subsubsection{求区间第 k 小(主席树)}
	给定 $n$ 个整数构成的序列 $a$,将对于指定的闭区间 $[l,r]$,查询其区间内的第 $k$ 小值。
	\begin{lstlisting}
#include<iostream>
#include<cstdio>
#include<algorithm>
using namespace std; 
const int N=2e5+7;
struct PST{
	int ls,rs;
	int num;
}t[N*20];
int a[N],b[N];
int rt[N],cnt;
void pushup(int o){
	t[o].num=t[t[o].ls].num+t[t[o].rs].num;
}
void build(int &o,int l,int r){
	o=++cnt;
	if(l==r){
		t[o].num=0;
		return;
	}
	int mid=(l+r)>>1;
	build(t[o].ls,l,mid);build(t[o].rs,mid+1,r);
	pushup(o);
}
int clone(int x){
	int r=++cnt;
	t[r]=t[x];
	return r;
}
void update(int &o,int l,int r,int c){
	o=clone(o);
	if(l==r){
		t[o].num++;
		return;
	}
	int mid=(l+r)>>1;
	if(c<=mid) update(t[o].ls,l,mid,c);
	else update(t[o].rs,mid+1,r,c);
	pushup(o);
}
int query(int ox,int oy,int l,int r,int k){
	if(l>=r) return l;
	int mid=(l+r)>>1;
	int z=t[t[oy].ls].num-t[t[ox].ls].num;
	if(z>=k) return query(t[ox].ls,t[oy].ls,l,mid,k);
	else return query(t[ox].rs,t[oy].rs,mid+1,r,k-z);
}
int main(){
	int i,j,k,m,n,x,y;
	scanf("%d%d",&n,&m);
	for(i=1;i<=n;i++){
		scanf("%d",&a[i]);
		b[i]=a[i];
	}
	sort(b+1,b+n+1);
	int d=unique(b+1,b+n+1)-b-1;
	build(rt[0],1,d);
	for(i=1;i<=n;i++){
		x=lower_bound(b+1,b+d+1,a[i])-b;
		rt[i]=rt[i-1];
		update(rt[i],1,d,x);
	}
	for(i=1;i<=m;i++){
		scanf("%d%d%d",&x,&y,&k);
		printf("%d\n",b[query(rt[x-1],rt[y],1,d,k)]);
	}
	return 0;
}\end{lstlisting}

\subsubsection{树上点权第 k 小}
	给定一棵 $n$ 个节点的树,每个点有一个权值。有 $m$ 个询问,每次给你 $u,v,k$,你需要回答 $\text{u xor last}$ 和 $v$ 这两个节点间第 $k$ 小的点权。
	
	其中 $\text{last}$ 是上一个询问的答案,定义其初始为 $0$,即第一个询问的 $u$ 是明文。
	\begin{lstlisting}
#include<iostream>
#include<cstdio>
#include<algorithm>
#define lson t[o].ls
#define rson t[o].rs
using namespace std;
const int N=1e5+7;
int a[N],b[N];
struct Graph{
	int to,nxt;
}e[N<<1];
int last[N],ecnt,f[N][20];
void addEdge(int u,int v){
	e[++ecnt].to=v;
	e[ecnt].nxt=last[u];
	last[u]=ecnt;
}
struct PST{
	int ls,rs,num;
}t[N*30];
int cnt,d; int rt[N], dep[N], lg[N];
void pushup(int o){
	t[o].num=t[lson].num+t[rson].num;
}
void build(int &o,int l,int r){
	o=++cnt;
	if(l==r){
		t[o].num=0; return;
	}
	int mid=(l+r)>>1;
	build(lson,l,mid); build(rson,mid+1,r);
}
int clone(int x){
	int r=++cnt; t[r]=t[x];
	return r;
}
void update(int &o,int l,int r,int c){
	o=clone(o);
	if(l==r){
		t[o].num++; return;
	}
	int mid=(l+r)>>1;
	if(c<=mid) update(lson,l,mid,c);
	else update(rson,mid+1,r,c);
	pushup(o);
}
void dfs(int fa,int x){
	f[x][0]=fa;
	for(int i=1;i<=18;i++) f[x][i]=f[f[x][i-1]][i-1];
	dep[x]=dep[fa]+1;
	rt[x]=rt[fa];
	update(rt[x],1,d,lower_bound(b+1,b+d+1,a[x])-b);
	int g=last[x];
	while(g){
		int v=e[g].to;
		if(v!=fa) dfs(x,v);
		g=e[g].nxt;
	}
}
int lca(int u,int v){
	if(dep[u]<dep[v]){
		int tmp=u;u=v;v=tmp;
	}
	while(dep[u]!=dep[v]){
		u=f[u][lg[dep[u]-dep[v]]];
	}
	if(u==v) return u;
	for(int i=18;i>=0;i--){
		if(f[u][i]!=f[v][i]){
			u=f[u][i]; v=f[v][i];
		}
	}
	return f[u][0];
}
int query(int ox,int oy,int oz,int of,int l,int r,int k){
	if(l==r) return l;
	int p=t[t[ox].ls].num+t[t[oy].ls].num-t[t[oz].ls].num-t[t[of].ls].num;
	int mid=(l+r)>>1;
	if(p>=k) return query(t[ox].ls,t[oy].ls,t[oz].ls,t[of].ls,l,mid,k);
	else return query(t[ox].rs,t[oy].rs,t[oz].rs,t[of].rs,mid+1,r,k-p);
}
int main(){
	int i,j,k,m,n,x,y,ls=0;
	scanf("%d%d",&n,&m);
	for(i=2;i<=n;i++) lg[i]=lg[i>>1]+1;
	for(i=1;i<=n;i++){
		scanf("%d",&a[i]);
		b[i]=a[i];
	}
	sort(b+1,b+n+1);
	d=unique(b+1,b+n+1)-b-1;
	build(rt[0],1,d);
	for(i=1;i<n;i++){
		scanf("%d%d",&x,&y);
		addEdge(x,y);addEdge(y,x);
	}
	dfs(0,1);
	for(i=1;i<=m;i++){
		scanf("%d%d%d",&x,&y,&k);
		x=x^ls;
		int la=lca(x,y);
		ls=b[query(rt[x],rt[y],rt[la],rt[f[la][0]],1,d,k)];
		printf("%d\n",ls);
	}
	return 0;
}\end{lstlisting}
	
\subsection{可持久化Trie树}
\subsubsection{求带权树最大异或路径}
	给定一棵 $n$ 个点的带权树,结点下标从 $1$ 开始到 $N$。寻找树中找两个结点,求最长的异或路径。
	异或路径指的是指两个结点之间唯一路径上的所有边权的异或。
	\begin{lstlisting}
#include<iostream>
#include<cstdio>
using namespace std;
int trie[5000000][2],cnt;
struct TU{
	int to,nxt,w;
}t[200005];
int ecnt,last[100005],ans=0;
void add(int u,int v,int w){
	t[++ecnt].to=v;
	t[ecnt].nxt=last[u];
	t[ecnt].w=w; 
	last[u]=ecnt;
}
void insert(int x){
	int i,u=0,v;
	for(i=30;i>=0;i--){
		v=(1<<i)&x?1:0;
		if(trie[u][v]==0){
			trie[u][v]=++cnt;
		}
		u=trie[u][v];
	}
}
int query(int x){
	int res=0,u=0,i,v;
	for(i=30;i>=0;i--){
		v=x&(1<<i)?0:1;
		if(trie[u][v]){
			res|=1<<i;
			u=trie[u][v];
		}
		else u=trie[u][1-v];
	}
	return res;
}
void dfs(int fa,int u,int nw){
	int e=last[u];
	while(e){
		int v=t[e].to;
		if(v==fa){
			e=t[e].nxt;
			continue;
		}
		int yw=nw^t[e].w;
		ans=max(ans,query(yw));
		insert(yw);
		dfs(u,v,yw);
		e=t[e].nxt;
	}
}
int main(){
	int i,j,k,m,n,u,v,w;
	scanf("%d",&n);
	for(i=1;i<n;i++){
		scanf("%d%d%d",&u,&v,&w);
		add(u,v,w);add(v,u,w);
	}
	insert(0);
	dfs(0,1,0);
	printf("%d\n",ans);
	return 0;
}\end{lstlisting}

\subsubsection{查询子树/路径上权值与给定值异或最大值}
	现在有一颗以 1 为根节点的由 n 个节点组成的树,节点从 1 至 n 编号。树上每个节点上都有一个权值 $v_i$。现在有 q 次操作,操作如下:
	\begin{itemize}
		\item \text{1 x z}:查询节点 x 的子树中的节点权值与 z 异或结果的最大值。
		\item \text{2 x y z}:查询节点 x 到节点 y 的简单路径上的节点的权值与 z 异或结果最大值。
	\end{itemize}
	\begin{lstlisting}
#include<iostream>
#include<cstdio>
using namespace std;
const int N=1e5+7;
struct Graph{
	int nxt,to;
}t[N<<1];
int val[N];
int last[N],ecnt;
void build(int u,int v){
	t[++ecnt].to=v;
	t[ecnt].nxt=last[u];
	last[u]=ecnt;
}
int dep[N],f[N][31],bl[N],br[N],lg[N],ct;
int trie[N*70][2],rt1[N],rt2[N],sze[N*70], cnt;
int clone(int o){
	int r=++cnt;
	trie[r][0]=trie[o][0];
	trie[r][1]=trie[o][1];
	sze[r]=sze[o];
	return r;
}
void inst(int &o,int x,int b){
	o=clone(o);sze[o]++;
	if(b==-1) return;
	int v=(x>>b)&1;
	inst(trie[o][v],x,b-1);
}
void dfs(int fa,int u){
	bl[u]=++ct;dep[u]=dep[fa]+1;
	f[u][0]=fa;
	int i;
	for(i=1;i<=17;i++) f[u][i]=f[f[u][i-1]][i-1];
	rt2[u]=rt2[fa];
	inst(rt2[u],val[u],30);
	rt1[ct]=rt1[ct-1];
	inst(rt1[ct],val[u],30);
	int e=last[u];
	while(e){
		int v=t[e].to;
		if(v==fa){
			e=t[e].nxt;
			continue;
		}
		dfs(u,v);
		e=t[e].nxt;
	}
	br[u]=ct;
}
int lca(int x,int y){
	int i;
	if(dep[x]<dep[y]){
		int tmp=x;x=y;y=tmp;
	}
	while(dep[x]!=dep[y]) x=f[x][lg[dep[x]-dep[y]]];
	if(x==y) return x;
	for(i=17;i>=0;i--){
		if(f[x][i]!=f[y][i]){
			x=f[x][i]; y=f[y][i];
		}
	}
	return f[x][0];
}
int query1(int o,int x){
	int res=0;
	int i,ul=rt1[bl[o]-1],ur=rt1[br[o]],v;
	for(i=30;i>=0;i--){
		v=((x>>i)&1)^1;
		if(sze[trie[ur][v]]-sze[trie[ul][v]]){
			res|=1<<i;
			ur=trie[ur][v];
			ul=trie[ul][v];
		}
		else{
			ur=trie[ur][v^1];
			ul=trie[ul][v^1];
		}
	}
	return res;
}
int query2(int ox,int oy,int x){
	int res=0;
	int i,ol=lca(ox,oy),v;
	int ux=rt2[ox],uy=rt2[oy],ul=rt2[ol],uf=rt2[f[ol][0]];
	for(i=30;i>=0;i--){
		v=((x>>i)&1)^1;
		if(sze[trie[ux][v]]+sze[trie[uy][v]]-sze[trie[ul][v]]-sze[trie[uf][v]]){
			res|=1<<i;
		}
		else{
			v^=1;
		}
		ux=trie[ux][v]; uy=trie[uy][v];
		ul=trie[ul][v]; uf=trie[uf][v];
	}
	return res;
}
int main(){
	int i,j,k,m,n,q,u,v,opt,x,y,z;
	scanf("%d%d",&n,&q);
	for(i=2;i<=n;i++) lg[i]=lg[i>>1]+1;
	for(i=1;i<=n;i++) scanf("%d",&val[i]);
	for(i=1;i<n;i++){
		scanf("%d%d",&u,&v);
		build(u,v);build(v,u);
	}
	dfs(0,1);
	for(i=1;i<=q;i++){
		scanf("%d",&opt);
		if(opt==1){
			scanf("%d%d",&x,&z);
			printf("%d\n",query1(x,z));
		}
		else{
			scanf("%d%d%d",&x,&y,&z);
			printf("%d\n",query2(x,y,z));
		}
	}
	return 0; 
}\end{lstlisting}
\subsection{可持久化并查集}
	给定 n 个集合,第 i 个集合内初始状态下只有一个数,为 i 。
	有 m 次操作。操作分为 3 种:
	\begin{itemize}
		\item 1 a b: 合并 a,b 所在集合;
		\item 2 k: 回到第 k 次操作(执行三种操作中的任意一种都记为一次操作)之后的状态;
		\item 3 a b: 询问 a,b 是否属于同一集合,如果是则输出 1 ,否则输出 0。
	\end{itemize}
	\begin{lstlisting}
#include <iostream>
#include <cstdio>
using namespace std;
const int N = 1e5 + 7;
int cnt = 0;
struct ZXS{
    int lson, rson, fa, dep;
}t[N * 40];
int rt[2 * N];
int n;
void build(int &o, int l, int r){
    o = ++ cnt;
    if(l == r){
        t[o].fa = l;
        return;
    }
    int mid = (l + r) >> 1;
    build(t[o].lson, l, mid);
    build(t[o].rson, mid + 1, r);
}
int clone(int o){
    int y = ++cnt;
    t[y] = t[o];
    return y;
}
void update(int &o, int l, int r, int x, int y){
    o = clone(o);
    if(l == r){
        t[o].fa = y;
        return;
    }
    int mid = (l + r) >> 1;
    if(x <= mid) update(t[o].lson, l, mid, x, y);
    else update(t[o].rson, mid + 1, r, x, y);
}
void add(int &o, int l, int r, int x){
    if(l == r){
        t[o].dep++;
        return;
    }
    int mid = (l + r) >> 1;
    if(x <= mid) add(t[o].lson, l, mid, x);
    else add(t[o].rson, mid + 1, r, x);
}
int query_fa(int o, int l, int r, int x){
    if(l == r){
        return o;
    }
    int mid = (l + r) >> 1;
    if(x <= mid) return query_fa(t[o].lson, l, mid, x);
    else return query_fa(t[o].rson, mid + 1, r, x);
}
int find(int x, int nw){
    int y = query_fa(rt[nw], 1, n, x);
    while(x != t[y].fa){
        x = t[y].fa;
        y = query_fa(rt[nw], 1, n, x);
    }
    return y;
}
int main(){
    int m, opt, x, y;
    scanf("%d%d", &n, &m);
    build(rt[0], 1, n);
    for(int i = 1; i <= m; i++){
        scanf("%d", &opt);
        rt[i] = rt[i - 1];
        if(opt == 1){
            scanf("%d%d", &x, &y);
            int xx = find(x, i), yy = find(y, i);
            if(t[xx].dep > t[yy].dep) swap(xx, yy);
            if(xx == yy) continue;
            update(rt[i], 1, n, t[xx].fa, t[yy].fa);
            if(t[xx].dep == t[yy].dep){
                int tmp = rt[i];
                add(tmp, 1, n, t[yy].fa);
            }
        }
        else if(opt == 2){
            scanf("%d", &x);
            rt[i] = rt[x];
        }
        else if(opt == 3){
            scanf("%d%d", &x, &y);
            if(t[find(x, i)].fa == t[find(y, i)].fa){
				printf("1\n");
			}
            else{
            	printf("0\n");
			}
        }
    }
    return 0;
}\end{lstlisting}
	