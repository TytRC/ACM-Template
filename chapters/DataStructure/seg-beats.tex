\section{区间最值操作和区间历史最值}
\subsection{区间最值}
笼统地说,区间最值操作指,将区间 $[l,r]$ 的数全部对 $x$ 取 $\max$ 或 $\min$,即 $a_i=\max(a_i, x)$ 或者 $a_i=min(a_i,x)$。下面是一道例题。\par 
	\href{https://acm.hdu.edu.cn/showproblem.php?pid=5306}{HDU5306 Gorgeous Sequence} \par 
	维护一个序列 $a$:
	\begin{enumerate}
		\item 0 l r t:  $\forall l \le i \le r, a_i = min(a_i, t)$。
		\item 1 l r: 输出区间 $[l, r]$ 中的最大值。
		\item 2 l r: 输出区间 $[l, r]$ 和。
		\item 多组数据,$T \le 100$,$n \le 10^6$, $\sum{m} \le 10^6$。
	\end{enumerate}
	区间取 $\min$,意味着只对那些大于 $t$ 的数有更改。因此这个操作的对象不再是整个区间,而是“这个区间中大于 $t$ 的数”。于是我们可以有这样的思路:每个结点维护该区间的最大值 $Max$、次大值 $Se$、区间和 $Sum$ 以及最大值的个数 $Cnt$。接下来我们考虑区间对 $t$ 取 $\min$ 的操作。

	如果 $Max \le t$,显然这个 $t$ 是没有意义的,直接返回;
	如果 $Se < t \le Max$,那么这个 $t$ 就能更新当前区间中的最大值。于是我们让区间和加上 $Cnt(t - Max)$,然后更新 $Max$ 为 $t$,并打一个标记。
	如果 $t \le Se$,那么这时你发现你不知道有多少个数涉及到更新的问题。于是我们的策略就是,暴力递归向下操作。然后上传信息。
	这个算法的复杂度如何?使用势能分析法可以得到复杂度是 $O(m \log n)$ 的。具体分析过程见论文。
	\lstinputlisting{./source/hdu5306.cpp}
	
	后续内容待补……