\section{树的重心}
\begin{definition}{树的重心}{tree-centroid}
	对于树上的每一个点,计算其所有子树中最大的子树节点数,这个值最小的点就是这棵树的重心。
	(这里以及下文中的“子树”都是指无根树的子树,即包括“向上”的那棵子树,并且不包括整棵树自身。)
\end{definition}

\subsection{性质}
\begin{itemize}
	\item 以树的重心为根时,所有子树的大小都不超过整棵树大小的一半。
	\item 树中所有点到某个点的距离和中,到重心的距离和是最小的;如果有两个重心,那么到它们的距离和一样。
	\item 把两棵树通过一条边相连得到一棵新的树,那么新的树的重心在连接原来两棵树的重心的路径上。
	\item 在一棵树上添加或删除一个叶子,那么它的重心最多只移动一条边的距离。
\end{itemize}

\subsection{求法}
在 DFS 中计算每个子树的大小,记录“向下”的子树的最大大小,利用总点数 - 当前子树(这里的子树指有根树的子树)的大小得到“向上”的子树的大小,然后就可以依据定义找到重心了。

\begin{lstlisting}
// 这份代码默认节点编号从 1 开始,即 i ∈ [1,n]
int size[MAXN],  // 这个节点的“大小”(所有子树上节点数 + 该节点)
    weight[MAXN],  // 这个节点的“重量”
    centroid[2];   // 用于记录树的重心(存的是节点编号)
void GetCentroid(int cur, int fa) {  // cur 表示当前节点 (current)
  size[cur] = 1;
  weight[cur] = 0;
  for (int i = head[cur]; i != -1; i = e[i].nxt) {
    if (e[i].to != fa) {  // e[i].to 表示这条有向边所通向的节点。
      GetCentroid(e[i].to, cur);
      size[cur] += size[e[i].to];
      weight[cur] = max(weight[cur], size[e[i].to]);
    }
  }
  weight[cur] = max(weight[cur], n - size[cur]);
  if (weight[cur] <= n / 2) {  // 依照树的重心的定义统计
    centroid[centroid[0] != 0] = cur;
  }
}\end{lstlisting}