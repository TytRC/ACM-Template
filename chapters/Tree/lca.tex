\section{最近公共祖先}
\subsection{倍增(略)}
\subsection{离线Tarjan}
\subsubsection{个人简略版}
\begin{lstlisting}
int ff[N];
int find2(int x){
    if(ff[x] != x) ff[x] = find2(ff[x]);
    return ff[x];
}
bool vis[N];
void tarjan(int x){
    ff[x] = x;
    vis[x] = 1;
    for(auto v: e[x]){
        tarjan(v);
        ff[v] = x;
    }
    for(auto v: qry[x]){
        if(vis[v]){
            int lca = find2(v); //本次询问x与v的lca
        }
    }
}
\end{lstlisting}
\subsubsection{oiwiki实现}
\lstinputlisting{./source/lca-tarjan-oiwiki.cpp}

\subsection{用欧拉序列转化为 RMQ 问题}

对一棵树进行 DFS,无论是第一次访问还是回溯,每次到达一个结点时都将编号记录下来,可以得到一个长度为 $2n - 1$ 的序列,这个序列被称作这棵树的欧拉序列。

在下文中,把结点 $u$ 在欧拉序列中第一次出现位置编号记为 $pos(u)$(也称作节点 $u$ 的欧拉序),把欧拉序列本身记作 $E[1..2n-1]$ 。

有了欧拉序列,LCA 问题可以在线性时间内转化为 RMQ 问题,即 $pos(LCA(u,v)) = min\{pos(k)|k \in E[pos(u)..pos(v)]\}$。

这个等式不难理解:从 $u$ 走到 $v$ 的过程中一定会经过 $LCA(u,v)$,但不会经过 $LCA(u,v)$ 的祖先。因此,从 $u$ 走到 $v$ 的过程中经过的欧拉序最小的结点就是 $LCA(u,v)$。

用 DFS 计算欧拉序列的时间复杂度是 $O(n)$,且欧拉序列的长度也是 $O(n)$,所以 LCA 问题可以在 $O(n)$ 的时间内转化成等规模的 RMQ 问题。

\subsubsection{个人实现}
\begin{lstlisting}
int ff[N];
int dfn[N * 2], pos[N * 2], st[20][N * 2], lg[N * 2], dep[N * 2], tot;
void dfs(int x, int fa, int d){
    ff[x] = fa;
    pos[x] = ++tot;
    dfn[tot] = x;
    dep[tot] = d;
    for(auto v: e[x]){
        dfs(v, x, d + 1);
        dfn[++tot] = x;
        dep[tot] = d;
    }
}
void st_pre(){
    for(int i = 1; i <= tot; i++) st[0][i] = i;
    for(int j = 1; j <= lg[tot]; j++){
        for(int i = 1; i + (1 << j) - 1 <= tot; i++){
            st[j][i] = dep[st[j - 1][i]] < dep[st[j - 1][i + (1 << (j - 1))]] ? 
                st[j - 1][i] : st[j - 1][i + (1 << (j - 1))];
        }
    }
}
int get(int x, int y){ //查询x,y两点的lca
    int l = pos[x], r = pos[y];
    if(l > r) swap(l, r);
    int k = lg[r - l + 1];
    int res = dep[st[k][l]] < dep[st[k][r - (1 << k) + 1]] ? st[k][l] : st[k][r - (1 << k) + 1];
    return dfn[res];
}
\end{lstlisting}
\subsubsection{oiwiki实现}
\begin{lstlisting}
int dfn[N << 1], dep[N << 1], dfntot = 0;
void dfs(int t, int depth) {
    dfn[++dfntot] = t;
    pos[t] = dfntot;
    dep[dfntot] = depth;
    for (int i = head[t]; i; i = side[i].next) {
        dfs(side[i].to, t, depth + 1);
        dfn[++dfntot] = t;
        dep[dfntot] = depth;
    }
}
void st_preprocess() {
    lg[0] = -1;  // 预处理 lg 代替库函数 log2 来优化常数
    for (int i = 1; i <= (N << 1); ++i) lg[i] = lg[i >> 1] + 1;
    for (int i = 1; i <= (N << 1) - 1; ++i) st[0][i] = dfn[i];
    for (int i = 1; i <= lg[(N << 1) - 1]; ++i)
        for (int j = 1; j + (1 << i) - 1 <= ((N << 1) - 1); ++j)
            st[i][j] = dep[st[i - 1][j]] < dep[st[i - 1][j + (1 << i - 1)]
                                               ? st[i - 1][j]
                                               : st[i - 1][j + (1 << i - 1)];
}
\end{lstlisting}

\subsection{树链剖分(略)}