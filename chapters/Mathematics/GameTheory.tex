\section{博弈论}
\subsection{Nim博弈}
$n$堆物品,每堆有$a_i$个,两个玩家轮流取走任意一堆的任意个物品,但不能不取。规定取走最后一个物品的人获胜。\par
\begin{definition}{Nim和}{nim}
    Nim和$=a_1\oplus a_2\oplus \dots\oplus a_n$
\end{definition}
\begin{definition}{mex函数}{mex}
    $mex(S)=\min\{x\}(x\notin S,x\in N)$
\end{definition}
\begin{definition}{SG函数}{sg}
    对于状态$x$和它的所有$k$个后继状态$y_1,y_2,\dots,y_k$,定义SG函数:
    $$SG(x)=mex\{SG(y_1),SG(y_2),...,SG(y_k)\}$$
\end{definition}
\subsection{Muti-Nim博弈}
\begin{theorem}{SG定理}{sg}
    游戏和的SG函数等于各个游戏SG函数的Nim和。
\end{theorem}
\subsection{Anti-Nim博弈}
在Nim博弈的条件下,规定拿走最后一枚石子的人输。
\begin{theorem}{SJ定理}{sj}
取最后一步的玩家判输,其余规则与一般SG游戏相同。 对 于任意一个Anti-SG游戏,如果定义所有子游戏SG值为0时游戏结束,先手必胜条件:
\begin{enumerate}
    \item 游戏的SG值为0且所有子游戏的SG值均不超过1
    \item 游戏的SG值不为0且至少一个子游戏的SG值超过1
\end{enumerate}
\end{theorem}