\section{多项式}
\subsection{快速傅里叶变换}
\begin{lstlisting}
// FFT.cpp
// Created by RainCurtain on 2020/11/24.
//

#include <iostream>
#include <cmath>
#include <cstring>
using namespace std;
const int M = 2e5 + 5;
const double PI = acos(-1);

struct Complex{
    double x, y;
    Complex(double x = 0, double y = 0):x(x),y(y){}
    Complex operator + (const Complex& t) const{
        return {x + t.x, y + t.y};
    }
    Complex operator - (const Complex& t) const{
        return {x - t.x, y - t.y};
    }
    Complex operator * (const Complex& t) const{
        return {x * t.x - y * t.y, x * t.y + y * t.x};
    }
};

int inv[M];
inline void change(Complex f[], int len){
    for (int i = 1; i < len; ++i){
        inv[i] = inv[i >> 1] >> 1;
        if (i & 1) inv[i] |= len / 2;
    }
    for (int i = 0; i < len; ++i){
        if (i < inv[i])
            swap(f[i], f[inv[i]]);
    }
}

inline void fft(Complex f[], int len, int on = 1){
    change(f, len);
    for (int h = 2; h <= len; h <<= 1){
        Complex wn(cos(2 * PI / h), sin(on * 2 * PI / h));
        for (int j = 0; j < len; j += h){
            Complex w(1);
            for (int k = j; k < j + h / 2; ++k){
                Complex u = f[k];
                Complex v = w * f[k + h / 2];
                f[k] = u + v;
                f[k + h / 2] = u - v;
                w = w * wn;
            }
        }
    }
    if (on == -1){
        for (int i = 0; i < len; ++i)
            f[i].x /= len, f[i].y /= len;
    }
}

Complex a[M], b[M];
char s1[M], s2[M];
int res[M];
int main(){
    while (~scanf("%s%s", s1, s2)){
        int len1 = strlen(s1);
        int len2 = strlen(s2);
        int len = 1;
        while (len < len1 * 2 || len < len2 * 2) len <<= 1;
        for (int i = 0; i < len1; ++i) a[i] = Complex(s1[len1 - i - 1] - '0');
        for (int i = len1; i < len; ++i) a[i] = {0, 0};
        for (int i = 0; i < len2; ++i) b[i] = Complex(s2[len2 - i - 1] - '0');
        for (int i = len2; i < len; ++i) b[i] = {0, 0};
        fft(a, len);
        fft(b, len);
        for (int i = 0; i < len; ++i) a[i] = a[i] * b[i];
        fft(a, len, -1);
        for (int i = 0; i < len; ++i) {
            res[i] = a[i].x + 0.5;
        }
        for (int i = 0; i < len; ++i) {
            res[i + 1] += res[i] / 10;
            res[i] %= 10;
        }
        len = len1 + len2 - 1;
        while (!res[len] && len) --len;
        for (int i = len; i >= 0; --i)
            printf("%d", res[i]);
        printf("\n");
    }
    return 0;
}
\end{lstlisting}
\subsection{快速数论变换}
\begin{lstlisting}
const int g = 3;//模数的原根
const ll mod = 998244353;//通常情况下的模数

int pow(int x,int y){
	ll z = 1ll * x, ans = 1ll;
    for (; y; y /= 2, z = z * z % mod)
    if (y & 1) ans = ans * z % mod;//注意精度
	return (int)ans % mod;
}
inline void ntt(int a[],int len,int inv){
	//前面和FFT一样
	for (int mid = 1; mid < len; mid *= 2){
		int tmp = pow(g, (mod - 1) / (mid * 2));//原根代替单位根
		if (inv == -1) tmp = pow(tmp, mod - 2);//逆变换则乘上逆元
		for (int i = 0; i < len; i += mid * 2){
			int omega = 1;
			for (ll j = 0; j < mid; ++j, omega = omega * tmp % mod){
				int x=a[i + j], y = omega * a[i + j + mid] % mod;
				a[i + j] = (x + y) % mod, a[i + j+ mid] = (x - y + mod) % mod;//注意取模
			}
		}//大体和FFT差不多
	}
}
\end{lstlisting}

\subsection{多项式全家桶}
qaq全能吧应该是
\begin{lstlisting}
#include<bits/stdc++.h>
using namespace std;
const int MAXN = 1 << 22;
typedef long long ll;
const ll g = 3, gi = 332748118;
const ll mod = 998244353;
int rev[MAXN];

ll fastpow(ll a, ll b) {
	ll res = 1;
	for(;b;b >>= 1) {
		if(b & 1) res = res * a % mod;
		a = (a * a) % mod;
	}
	return res % mod;
}
void change(ll a[], int len) {
	for(int i = 0; i < len; i++) {
		rev[i] = rev[i >> 1] >> 1;
		if(i & 1) {
			rev[i] |= len >> 1;
		}
	}
	for(int i = 0; i < len; i++) if(i > rev[i]) swap(a[i], a[rev[i]]);
}
void NTT(ll a[], int len, int on) {
	change(a, len);
	for(int l = 2; l <= len; l <<= 1) {
		ll step = fastpow(on == 1 ? g : gi, (mod - 1) / l);
		for(int i = 0; i < len; i += l) {
			ll w = 1;
			for(int j = i; j < i + l / 2; j++) {
				ll g = a[j], h = w * a[j + l / 2] % mod;
				a[j] = (g + h) % mod; a[j + l / 2] = (g - h + mod) % mod;
				w = w * step % mod;
			}
		}
	}
	ll inv = fastpow(len, mod - 2);
	if(on == -1) for(int i = 0; i < len; i++) a[i] = a[i] * inv % mod;
}

const double PI = acos(-1);
struct Complex {
	double x, y;
	Complex(double _x = 0.0, double _y = 0.0) : x(_x), y(_y){
	}
	Complex operator + (const Complex  & rhs) const{ return Complex(x + rhs.x, y + rhs.y);}
	Complex operator - (const Complex  & rhs) const{ return Complex(x - rhs.x, y - rhs.y);}
	Complex operator * (const Complex  & rhs) const{ return Complex(x * rhs.x - y * rhs.y, y * rhs.x + x * rhs.y);}
};
void change(Complex a[], int len) {
	for(int i = 0; i < len; i++) {
		rev[i] = rev[i >> 1] >> 1;
		if(i & 1) {
			rev[i] |= len >> 1;
		}
	}
	for(int i = 0; i < len; i++) if(i > rev[i]) swap(a[i], a[rev[i]]);
}
void FFT(Complex a[], int len, int on) {
	change(a, len);
	for(int l = 2; l <= len; l <<= 1) {
		Complex step = Complex(cos(2.0 * PI / l), sin( on * 2.0 * PI / l));
		for(int i = 0; i < len; i += l) {
			Complex w = Complex(1, 0);
			for(int j = i; j < i + l / 2; j++) {
				Complex g = a[j], h = w * a[j + l / 2];
				a[j] = g + h; a[j + l / 2] = g - h;
				w = w * step;
			}
		}
	}
	if(on == -1) for(int i = 0; i < len; i++) a[i].x /= len;
}

void FWT_or(ll a[], int n, int type) {
	for(int l = 2, k = 1; l <= 1 << n; l <<= 1, k <<= 1) {
		for(int i = 0; i < 1 << n; i += l) {
			for(int j = 0; j < k; j++) {
				a[i + j + k] = (a[i + j] * type + a[i + j + k] + mod) % mod;
			}
		}
	}
}
void FWT_and(ll a[], int n, int type) {
	for(int l = 2, k = 1; l <= 1 << n; l <<= 1, k <<= 1) {
		for(int i = 0; i < 1 << n; i += l) {
			for(int j = 0; j < k; j++) {
				a[i + j] = (a[i + j] + type * a[i + j + k] + mod) % mod;
			}
		}
	}
}
void FWT(ll a[], int n, int type) {
	ll inv = fastpow(2, mod - 2);
	for(int l = 2, k = 1; l <= 1 << n; l <<= 1, k <<= 1) {
		for(int i = 0; i < 1 << n; i += l) {
			for(int j = 0; j < k; j++) {
				ll g = a[i + j], h = a[i + j + k];
				a[i + j] = (g  + h) % mod;
				a[i + j + k] = (g - h + mod) % mod;
				if(type == -1) {
					a[i + j] = a[i + j] * inv % mod;
					a[i + j + k]  = a[i + j + k] * inv % mod;
				}
			}
		}
	}
}

int limit;
ll Buffer[MAXN], Inv[MAXN], Der[MAXN], Ln[MAXN], Exp[MAXN], Int[MAXN];
void polyinv(ll h[], int n, ll f[]) {
	if(n == 1) {
		f[0] = fastpow(h[0], mod - 2);
		return;
	}
	polyinv(h, (n + 1) >> 1, f);
	for(limit = 1; limit <= n + n; limit <<= 1);
	copy(h, h + n, Buffer);
	fill(Buffer + n, Buffer + limit, 0);
	NTT(Buffer, limit, 1); NTT(f, limit, 1);
	for(int i = 0; i < limit; i++) {
		f[i] = f[i] * ((2 - f[i] * Buffer[i] % mod + mod) % mod) % mod;
	}
	NTT(f, limit, -1);
	fill(f + n, f + limit, 0);
}

void polydiv(ll f[], ll g[], int n, int m, ll q[], ll r[]) {
	fill(Inv, Inv + (max(n, m) << 2), 0);
	for(int i = 0; i <= n / 2; i++) {
		swap(f[i], f[n - i]);
	}
	for(int i = 0; i <= m / 2; i++) {
		swap(g[i], g[m - i]);
	}
	polyinv(g, n - m + 1, Inv);
	for(limit = 1; limit <= n * 2 - m + 2; limit <<= 1);
	NTT(Inv, limit, 1); NTT(f, limit, 1);
	for(int i = 0; i < limit; i++) {
		q[i] = Inv[i] * f[i] % mod;
	}
	NTT(q, limit, -1);
	NTT(f, limit, -1);
	for(int i = 0; i <= (n - m) / 2; i++) swap(q[i], q[n - m - i]);
	fill(q + n - m + 1, q + limit, 0);
	for(int i = 0; i <= n / 2; i++) {
		swap(f[i], f[n - i]);
	}
	for(int i = 0; i <= m / 2; i++) {
		swap(g[i], g[m - i]);
	}
	for(limit = 1; limit <= n; limit <<= 1);
	NTT(f, limit, 1); NTT(g, limit, 1); NTT(q, limit, 1);
	for(int i = 0; i < limit; i++) {
		r[i] = (f[i] - q[i] * g[i] % mod + mod) % mod;
	}
	NTT(r, limit, -1);
	NTT(q, limit, -1);
}

void polyder(ll h[], int n, ll f[]) {
	fill(f, f + n, 0);
	for(int i = 1; i < n; i++) {
		f[i - 1] = h[i] * (ll) i % mod;
	}
	f[n - 1] = 0;
}

void polyint(ll h[], int n, ll f[]) {
	fill(f, f + n + 1, 0);
	for(int i = 1; i < n; i++) {
		f[i] = h[i - 1] * fastpow((ll)i, mod - 2) % mod;
	}
	f[0] = 0;
}
void polyln(ll h[], int n, ll f[]) {
	fill(f, f + (n << 2), 0);
	fill(Inv, Inv + (n << 2), 0);
	fill(Der, Der + (n << 2), 0);
	polyinv(h, n, Inv);
	polyder(h, n, Der);
	for(limit = 1; limit <= n * 2; limit <<= 1);
	fill(Inv + n, Inv + limit, 0); fill(Der + n, Der + limit, 0);
	NTT(Inv, limit, 1); NTT(Der, limit, 1);
	for(int i = 0; i < limit; i++) {
		Der[i] = Der[i] * Inv[i] % mod;
	}
	NTT(Der, limit, -1);
	fill(Der + n, Der + limit, 0);
	polyint(Der, n, f);	
}
void polyexp(ll h[], int n, ll f[]) {
	if(n == 1) {
		f[0] = 1;
		return;
	}
	polyexp(h, (n + 1) >> 1, f);
	fill(Ln, Ln + (n << 2), 0);
	polyln(f, n, Ln);
	for(limit = 1; limit <= n + n; limit <<= 1);
	copy(h, h + n, Buffer);
	fill(Buffer + n, Buffer + limit, 0);
	NTT(Ln, limit, 1); NTT(Buffer, limit, 1); NTT(f, limit, 1);
	for(int i = 0; i < limit; i++) {
		f[i] = f[i] * ((1 - Ln[i] + Buffer[i] + mod) % mod) % mod;
	}
	NTT(f, limit, -1);
	fill(f + n, f + limit, 0);
}
int n, m; 
ll A[MAXN], B[MAXN], C[MAXN], D[MAXN], E[MAXN];
int main() {
	return 0;
}
\end{lstlisting}