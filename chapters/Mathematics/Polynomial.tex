\section{多项式}
\subsection{快速傅里叶变换}
\begin{lstlisting}
// FFT.cpp
// Created by RainCurtain on 2020/11/24.
//

#include <iostream>
#include <cmath>
#include <cstring>
using namespace std;
const int M = 2e5 + 5;
const double PI = acos(-1);

struct Complex{
    double x, y;
    Complex(double x = 0, double y = 0):x(x),y(y){}
    Complex operator + (const Complex& t) const{
        return {x + t.x, y + t.y};
    }
    Complex operator - (const Complex& t) const{
        return {x - t.x, y - t.y};
    }
    Complex operator * (const Complex& t) const{
        return {x * t.x - y * t.y, x * t.y + y * t.x};
    }
};

int inv[M];
inline void change(Complex f[], int len){
    for (int i = 1; i < len; ++i){
        inv[i] = inv[i >> 1] >> 1;
        if (i & 1) inv[i] |= len / 2;
    }
    for (int i = 0; i < len; ++i){
        if (i < inv[i])
            swap(f[i], f[inv[i]]);
    }
}

inline void fft(Complex f[], int len, int on = 1){
    change(f, len);
    for (int h = 2; h <= len; h <<= 1){
        Complex wn(cos(2 * PI / h), sin(on * 2 * PI / h));
        for (int j = 0; j < len; j += h){
            Complex w(1);
            for (int k = j; k < j + h / 2; ++k){
                Complex u = f[k];
                Complex v = w * f[k + h / 2];
                f[k] = u + v;
                f[k + h / 2] = u - v;
                w = w * wn;
            }
        }
    }
    if (on == -1){
        for (int i = 0; i < len; ++i)
            f[i].x /= len, f[i].y /= len;
    }
}

Complex a[M], b[M];
char s1[M], s2[M];
int res[M];
int main(){
    while (~scanf("%s%s", s1, s2)){
        int len1 = strlen(s1);
        int len2 = strlen(s2);
        int len = 1;
        while (len < len1 * 2 || len < len2 * 2) len <<= 1;
        for (int i = 0; i < len1; ++i) a[i] = Complex(s1[len1 - i - 1] - '0');
        for (int i = len1; i < len; ++i) a[i] = {0, 0};
        for (int i = 0; i < len2; ++i) b[i] = Complex(s2[len2 - i - 1] - '0');
        for (int i = len2; i < len; ++i) b[i] = {0, 0};
        fft(a, len);
        fft(b, len);
        for (int i = 0; i < len; ++i) a[i] = a[i] * b[i];
        fft(a, len, -1);
        for (int i = 0; i < len; ++i) {
            res[i] = a[i].x + 0.5;
        }
        for (int i = 0; i < len; ++i) {
            res[i + 1] += res[i] / 10;
            res[i] %= 10;
        }
        len = len1 + len2 - 1;
        while (!res[len] && len) --len;
        for (int i = len; i >= 0; --i)
            printf("%d", res[i]);
        printf("\n");
    }
    return 0;
}
\end{lstlisting}
\subsection{快速数论变换}
\begin{lstlisting}
const int g = 3;//模数的原根
const ll mod = 998244353;//通常情况下的模数

int pow(int x,int y){
	ll z = 1ll * x, ans = 1ll;
    for (; y; y /= 2, z = z * z % mod)
    if (y & 1) ans = ans * z % mod;//注意精度
	return (int)ans % mod;
}
inline void ntt(int a[],int len,int inv){
	//前面和FFT一样
	for (int mid = 1; mid < len; mid *= 2){
		int tmp = pow(g, (mod - 1) / (mid * 2));//原根代替单位根
		if (inv == -1) tmp = pow(tmp, mod - 2);//逆变换则乘上逆元
		for (int i = 0; i < len; i += mid * 2){
			int omega = 1;
			for (ll j = 0; j < mid; ++j, omega = omega * tmp % mod){
				int x=a[i + j], y = omega * a[i + j + mid] % mod;
				a[i + j] = (x + y) % mod, a[i + j+ mid] = (x - y + mod) % mod;//注意取模
			}
		}//大体和FFT差不多
	}
}
\end{lstlisting}

\subsection{多项式全家桶}
多项式除法|余数、求逆、exp、ln、开方
\begin{lstlisting}

#include<bits/stdc++.h>
using namespace std;

typedef long long ll;
const ll g = 3, gi = 332748118;
const ll mod = 998244353;

typedef vector<ll> Poly;

ll fastpow(ll a, ll b) {
	ll res = 1;
	for(;b;b >>= 1) {
		if(b & 1) res = res * a % mod;
		a = (a * a) % mod;
	}
	return res % mod;
}
void change(Poly & a, int len) {
	vector<int> rev(len);
	for(int i = 0; i < len; i++) {
		rev[i] = rev[i >> 1] >> 1;
		if(i & 1) {
			rev[i] |= len >> 1;
		}
	}
	for(int i = 0; i < len; i++) if(i > rev[i]) swap(a[i], a[rev[i]]);
}
void NTT(Poly & a, int len, int on) {
	change(a, len);
	for(int l = 2; l <= len; l <<= 1) {
		ll step = fastpow(on == 1 ? g : gi, (mod - 1) / l);
		for(int i = 0; i < len; i += l) {
			ll w = 1;
			for(int j = i; j < i + l / 2; j++) {
				ll g = a[j], h = w * a[j + l / 2] % mod;
				a[j] = (g + h) % mod; a[j + l / 2] = (g - h + mod) % mod;
				w = w * step % mod;
			}
		}
	}
	ll inv = fastpow(len, mod - 2);
	if(on == -1) for(int i = 0; i < len; i++) a[i] = a[i] * inv % mod;
}

Poly Mul(Poly a, Poly b) {	//多项式乘法
	int n = int(a.size()), m = int(b.size());
	int limit = 1;
	while(limit < m + n) limit <<= 1;
	a.resize(limit), b.resize(limit);
	NTT(a, limit, 1); NTT(b, limit, 1);
	for(int i = 0; i < limit; i++) a[i] = a[i] * b[i];
	NTT(a, limit, -1);
	a.resize(n + m - 1);
	return a;
}


Poly Inv(Poly a, int len) {	//多项式求逆 输入A(x) 返回 1 / A(x) mod  (x ^ len)
	a.resize(len);
	if(len == 1) {
		a[0] = fastpow(a[0], mod - 2);
		return a;
	}
	Poly g = Inv(a, (len + 1) >> 1);
	int limit = 1;
	while(limit < (len << 1)) limit <<= 1;
	g.resize(limit);
	a.resize(limit);
	NTT(g, limit, 1);
	NTT(a, limit, 1);
	for(int i = 0; i < limit; i++) {
		a[i] = (2 - g[i] * a[i] % mod + mod) % mod * g[i];
	}
	NTT(a, limit, -1);
	a.resize(len);
	return a;
}

Poly Sqrt(Poly a, int len) { // 多项式开方 输入A(x) 返回 sqrt(A(x)) mod (x ^ len)
	a.resize(len);
	if(len == 1) {
		a[0] = 1;
		return a;
	}
	Poly g = Sqrt(a, (len + 1) >> 1);
	Poly h = Inv(g, len);
	int limit = 1;
	while(limit < (len << 1)) limit <<= 1;
	g.resize(limit); a.resize(limit), h.resize(limit);
	NTT(g, limit, 1); NTT(a, limit, 1); NTT(h, limit, 1);
	for(int i = 0; i < limit; i++) a[i] = (g[i] * g[i] % mod + a[i]) % mod * (mod + 1) / 2 % mod * h[i] % mod;
	NTT(a, limit, -1);
	a.resize(len);
	return a;
}

Poly Der(Poly a, int len) {	//多项式求导
	a.resize(len);
	for(int i = 0; i < len - 1; i++) {
		a[i] = (i + 1) * a[i + 1];
	}
	a.resize(len - 1);
	return a;
}

Poly Int(Poly a, int len) {	//多项式积分
	
	a.resize(len);
	for(int i = len - 1; i > 0; i--) {
		a[i] = a[i - 1] * fastpow(i, mod - 2) % mod;
	}
	a[0] = 0;
	return a;
}

Poly Ln(Poly a, int len) {	//多项式ln 输入A(x) 返回 Ln A(x) mod (x ^ len)
	Poly d = Der(a, len), g = Inv(a, len);
	return Int(Mul(d, g), len);
}

Poly Exp(Poly a, int len) {	//多项式exp 输入A(x) 返回 e^A(x) mod (x ^ len)
	a.resize(len);
	if(len == 1) {
		a[0] = 1;
		return a;
	}
	Poly g = Exp(a, (len + 1) >> 1);
	Poly l = Ln(g, len);
	int limit = 1;
	while(limit < (len << 1)) limit <<= 1;
	g.resize(limit); l.resize(limit); a.resize(limit);
	NTT(a, limit, 1); NTT(l, limit, 1); NTT(g, limit, 1);
	for(int i = 0; i < limit; i++) a[i] = (1 - l[i] + a[i] + mod) % mod * g[i];
	NTT(a, limit, -1);
	a.resize(len);
	return a;
}

pair<Poly, Poly> Div(Poly f, Poly g) {	//多项式除法 输入Q(x) F(x) = Q(x)G(x) + R(x) 返回 {Q(x), R(x)}
	
	int n = f.size() - 1, m = g.size() - 1;		
	reverse(f.begin(), f.end()); reverse(g.begin(), g.end());
	
	Poly inv = Inv(g, f.size() - g.size() + 1);
	Poly q = Mul(f, inv);
	q.resize(n - m + 1);
	reverse(q.begin(), q.end()); reverse(f.begin(), f.end()); reverse(g.begin(), g.end());
	
	int limit = 1;
	for(limit = 1; limit <= n; limit <<= 1);
	
	f.resize(limit); g.resize(limit); q.resize(limit);
	NTT(f, limit, 1); NTT(g, limit, 1); NTT(q, limit, 1);
	
	Poly r(limit);
	
	for(int i = 0; i < limit; i++) {
		r[i] = (f[i] - q[i] * g[i] % mod + mod) % mod;
	}
	
	NTT(r, limit, -1); NTT(q, limit, -1);
	
	r.resize(m);
	q.resize(n - m + 1);
	return {q, r};
}

int main() {
	
	return 0;
}

\end{lstlisting}