\section{数论}
\subsection{Miller-Rabin素数测试}
\begin{lstlisting}
bool millerRabbin(int n) {
  if (n < 3) return n == 2;
  int a = n - 1, b = 0;
  while (a % 2 == 0) a /= 2, ++b;
  // test_time 为测试次数,建议设为不小于 8
  // 的整数以保证正确率,但也不宜过大,否则会影响效率
  for (int i = 1, j; i <= test_time; ++i) {
    int x = rand() % (n - 2) + 2, v = quickPow(x, a, n);
    if (v == 1 || v == n - 1) continue;
    for (j = 0; j < b; ++j) {
      v = (long long)v * v % n;
      if (v == n - 1) break;
    }
    if (j >= b) return 0;
  }
  return 1;
}
\end{lstlisting}
\subsection{线性筛}
\begin{lstlisting}
void init() {
  phi[1] = 1;
  mu[1] = 1;
  for (int i = 2; i < MAXN; ++i) {
    if (!vis[i]) {
      phi[i] = i - 1; mu[i] = -1;
      pri[++cnt] = i;
    }
    for (int j = 1; j <= cnt && 1LL * i * pri[j] < MAXN; ++j) {
      vis[i * pri[j]] = 1;
      if (i % pri[j]) {
        phi[i * pri[j]] = phi[i] * (pri[j] - 1);
        mu[i * pri[j]] = -mu[i];
      } else {
        mu[i * pri[j]] = 0;
        phi[i * pri[j]] = phi[i] * pri[j];
        break;
      }
    }
  }
}
\end{lstlisting}
\subsection{拓展欧几里得}
\begin{lstlisting}
int exgcd(int a, int b, int &x, int &y) {
  if (!b) {
    x = 1; y = 0;
    return a;
  }
  int g = exgcd(b, a % b, y, x);
  y -= (a / b) * x;
  return d;
}
\end{lstlisting}
\subsection{欧拉函数与欧拉定理}
欧拉函数具有以下的性质:
\begin{itemize}
  \item $\varphi = n\times\prod_{i=1}^np_i^{k_i}$
  \item $id=\sum_{d|n}\varphi(d)=\varphi*1$,所以$\varphi=\mu*id$
\end{itemize}
\textbf{拓展欧拉定理为:}
\begin{equation*}
  a^b\equiv
  \begin{cases}
  a^{b\bmod\varphi(p)},\,&\gcd(a,\,p)=1\\
  a^b,&\gcd(a,\,p)\ne1,\,b<\varphi(p)\\
  a^{b\bmod\varphi(p)+\varphi(p)},&\gcd(a,\,p)\ne1,\,b\ge\varphi(p)
  \end{cases}
  \pmod p
\end{equation*}
\subsection{乘法逆元}
\textbf{拓展欧几里得求乘法逆元:}略。\par
\textbf{快速幂求乘法逆元:}$ax\equiv 1\pmod b\text{,那么} x\equiv a^{b-2}\pmod b$
\subsubsection{线性求逆元}
线性( $O(n)$ )时间复杂度求出 $1,2,...,n$ 中每个数关于 $p$ 的逆元。\\
首先,很显然的 $1^{-1} \equiv 1 \pmod p$ ;\par
其次对于递归情况 $i^{-1}$ ,我们令 $k = \lfloor \frac{p}{i} \rfloor$ , $j = k \bmod i$ ,有 $p = ki + j$ 。再放到 $\mod p$ 意义下就会得到: $ki+j \equiv 0 \pmod p$ ;\par
两边同时乘 $i^{-1}j^{-1}$ :
\begin{align*}
  kj^{-1}+i^{-1} \equiv 0 \pmod p \\
  i^{-1} \equiv -kj^{-1} \pmod p
\end{align*}
再带入 $j = k \bmod i$ ,有 $p = ki + j$ ,有:\par
 $i^{-1} \equiv -\lfloor\frac{p}{i}\rfloor (p \bmod i)^{-1} \pmod p$ 

我们注意到 $p \bmod i < i$ ,而在迭代中我们完全可以假设我们已经知道了所有的模 $p$ 下的逆元 $j^{-1}, j < i$ 。

故我们就可以推出逆元,利用递归的形式,而使用迭代实现:
\begin{lstlisting}
inv[1] = 1;
for (int i = 2; i <= n; ++i) {
  inv[i] = (long long)(p - p / i) * inv[p % i] % p;
}
\end{lstlisting}
\subsubsection{线性求任意n个数的逆元}
上面的方法只能求 $1$ 到 $n$ 的逆元,如果需要求任意给定 $n$ 个数( $1 \le a_i < p$ )的逆元,就需要下面的方法:\par
首先计算 $n$ 个数的前缀积,记为 $s_i$ ,然后使用快速幂或扩展欧几里得法计算 $s_n$ 的逆元,记为 $sv_n$ 。\par
因为 $sv_n$ 是 $n$ 个数的积的逆元,所以当我们把它乘上 $a_n$ 时,就会和 $a_n$ 的逆元抵消,于是就得到了 $a_1$ 到 $a_{n-1}$ 的积逆元,记为 $sv_{n-1}$ 。\par
同理我们可以依次计算出所有的 $sv_i$ ,于是 $a_i^{-1}$ 就可以用 $s_{i-1} \times sv_i$ 求得。\par
所以我们就在 $O(n + \log p)$ 的时间内计算出了 $n$ 个数的逆元。
\begin{lstlisting}
s[0] = 1;
for (int i = 1; i <= n; ++i) s[i] = s[i - 1] * a[i] % p;
sv[n] = qpow(s[n], p - 2);
for (int i = n; i >= 1; --i) sv[i - 1] = sv[i] * a[i] % p;
for (int i = 1; i <= n; ++i) inv[i] = sv[i] * s[i - 1] % p;
\end{lstlisting}
\subsection{线性同余方程}
求解$ax\equiv c\pmod b$,即求解$ax+by=c$。
\begin{lstlisting}
bool liEu(int a, int b, int c, int& x, int& y) {
  int d = exgcd(a, b, x, y);
  if (c % d != 0) return false;
  int k = c / d;
  x *= k; y *= k;
  return true;
}
\end{lstlisting}
\subsection{中国剩余定理}
中国剩余定理 (Chinese Remainder Theorem, CRT) 可求解如下形式的一元线性同余方程组(其中 $n_1, n_2, \cdots, n_k$ 两两互质):
\begin{equation*}
  \begin{cases}
  x &\equiv a_1 \pmod {n_1} \\
  x &\equiv a_2 \pmod {n_2} \\
    &\vdots \\
  x &\equiv a_k \pmod {n_k} \\
  \end{cases}
\end{equation*}代码如下:
\begin{lstlisting}
const int MAXN = 1e5 + 5;
typedef long long ll;
ll w[MAXN], b[MAXN];
ll exgcd(ll _a, ll _b, ll& x, ll& y){}
ll mul(ll x, ll y, ll m){
  ll res = 0;
  int sig = 1;
  if (y < 0) y = -y, sig = -1;
  while (y){
    if (y & 1) res = (res + x) % m;
    x = (x + x) % m;
    y /= 2;
  }
  return res * sig;
}
ll excrt(){
  ll ans = w[1], p = b[1], x, y;
  for (int i = 2; i <= n; ++i) {
      ll g = exgcd(p, b[i], x, y);
      p = p * (b[i] / g);
      ans = (ans - w[i] % p + p) % p;
      y = mul(ans / g, y, p); // mul:龟速乘
      ans = ((w[i] + mul(b[i], y, p)) % p + p) % p;
  }
  return ans;
}
\end{lstlisting}
\subsection{卢卡斯定理}
Lucas 定理内容如下:对于质数 $p$ ,有
\begin{equation*}
  \binom{n}{m}\bmod p = \binom{\left\lfloor n/p \right\rfloor}{\left\lfloor m/p\right\rfloor}\cdot\binom{n\bmod p}{m\bmod p}\bmod p
\end{equation*}
观察上述表达式,可知 $n\bmod p$ 和 $m\bmod p$ 一定是小于 $p$ 的数,可以直接求解, $\displaystyle\binom{\left\lfloor n/p \right\rfloor}{\left\lfloor m/p\right\rfloor}$ 可以继续用 Lucas 定理求解。这也就要求 $p$ 的范围不能够太大,一般在 $10^5$ 左右。边界条件:当 $m=0$ 的时候,返回 $1$ 。\par
时间复杂度为 $O(f(p) + g(n)\log n)$ ,其中 $f(n)$ 为预处理组合数的复杂度, $g(n)$ 为单次求组合数的复杂度。
\begin{lstlisting}
long long Lucas(long long n, long long m, long long p) {
    if (m == 0) return 1;
    return (C(n % p, m % p, p) * Lucas(n / p, m / p, p)) % p;
}
\end{lstlisting}
\textbf{拓展卢卡斯定理}此时,不要求$p$是质数。
\begin{lstlisting}
LL CRT(int n, LL* a, LL* m) {
  LL M = 1, p = 0;
  for (int i = 1; i <= n; i++) M = M * m[i];
  for (int i = 1; i <= n; i++) {
    LL w = M / m[i], x, y;
    exgcd(w, m[i], x, y);
    p = (p + a[i] * w * x % mod) % mod;
  }
  return (p % mod + mod) % mod;
}
LL calc(LL n, LL x, LL P) {
  if (!n) return 1;
  LL s = 1;
  for (int i = 1; i <= P; i++)
    if (i % x) s = s * i % P;
  s = Pow(s, n / P, P);
  for (int i = n / P * P + 1; i <= n; i++)
    if (i % x) s = s * i % P;
  return s * calc(n / x, x, P) % P;
}
LL multilucas(LL m, LL n, LL x, LL P) {
  int cnt = 0;
  for (int i = m; i; i /= x) cnt += i / x;
  for (int i = n; i; i /= x) cnt -= i / x;
  for (int i = m - n; i; i /= x) cnt -= i / x;
  // int inverse(int x) 函数返回x在模p意义下的逆元
  return Pow(x, cnt, P) % P * calc(m, x, P) % P * inverse(calc(n, x, P), P) %
         P * inverse(calc(m - n, x, P), P) % P;
}
LL exlucas(LL m, LL n, LL P) {
  int cnt = 0;
  LL p[20], a[20];
  for (LL i = 2; i * i <= P; i++) {
    if (P % i == 0) {
      p[++cnt] = 1;
      while (P % i == 0) p[cnt] = p[cnt] * i, P /= i;
      a[cnt] = multilucas(m, n, i, p[cnt]);
    }
  }
  if (P > 1) p[++cnt] = P, a[cnt] = multilucas(m, n, P, P);
  return CRT(cnt, a, p);
}
\end{lstlisting}
\subsection{莫比乌斯反演}
常用Dirichlet卷积:
\begin{equation*}
  \begin{aligned}
    \varepsilon=\mu \ast 1&\iff\varepsilon(n)=\sum_{d\mid n}\mu(d)\\
    d=1 \ast 1&\iff d(n)=\sum_{d\mid n}1\\
    \sigma=\operatorname{id} \ast 1&\iff\sigma(n)=\sum_{d\mid n}d\\
    \varphi=\mu \ast \operatorname{id}&\iff\varphi(n)=\sum_{d\mid n}d\cdot\mu(\frac{n}{d})\\
    \varphi \ast 1=\operatorname{id}
    \end{aligned}
\end{equation*}\par
常用结论:
$d(i\cdot j)=\sum_{x \mid i}\sum_{y \mid j}[\gcd(x,y)=1]$
\begin{equation*}
\begin{aligned}
d(i\cdot j)=&\sum_{x \mid i}\sum_{y \mid j}[\gcd(x,y)=1]\\
=&\sum_{p \mid i,p \mid j}\mu(p)d\left(\frac{i}{p}\right)d\left(\frac{j}{p}\right)
\end{aligned}
\end{equation*}\par
$\mu$为莫比乌斯函数,是积性函数,并且满足
\begin{equation*}
  \sum_{d\mid n}\mu(d)=
  \begin{cases}
  1&n=1\\
  0&n\neq 1\\
  \end{cases}
\end{equation*}\par
\textbf{莫比乌斯反演}
\begin{theorem}{莫比乌斯反演}{mobius}
设 $f(n),g(n)$ 为两个数论函数。\\
如果有 $f(n)=\sum_{d\mid n}g(d)$ ,那么有 $g(n)=\sum_{d\mid n}\mu(d)f(\frac{n}{d})$ 。\\
如果有 $f(n)=\sum_{n|d}g(d)$ ,那么有 $g(n)=\sum_{n|d}\mu(\frac{d}{n})f(d)$ 。
\end{theorem}
\subsection{杜教筛}
\begin{theorem}{杜教筛}{dujiao}
对于任意一个数论函数 $g$ ,必满足
\begin{equation*}
  \sum_{i=1}^{n}\sum_{d \mid i}f(d)g\left(\frac{i}{d}\right)=\sum_{i=1}^{n}g(i)S\left(\left\lfloor\frac{n}{i}\right\rfloor\right)\\
\Leftrightarrow
\sum_{i=1}^{n}(f\ast g)(i)=\sum_{i=1}^{n}g(i)S\left(\left\lfloor\frac{n}{i}\right\rfloor\right)
\end{equation*}\par
\end{theorem}
\begin{note}
杜教筛需要预处理和记忆化搜索,否则会爆时间复杂度
\end{note}\par
由定理\ref{thm:dujiao}可以得到递推式
\begin{equation*}
  g(1)S(n)=\sum_{i=1}^n(f\ast g)(i)-\sum_{i=2}^ng(i)S\left(\left\lfloor\frac{n}{i}\right\rfloor\right)
\end{equation*}\par
\textbf{莫比乌斯函数前缀和}$\epsilon =\mu \ast 1$
\begin{align*}
S(n)=\sum_{i=1}^n \epsilon (i)-\sum_{i=2}^n S(\lfloor \frac n i \rfloor)= 1-\sum_{i=2}^n S(\lfloor \frac n i \rfloor)
\end{align*}
\begin{lstlisting}
ll get_mu(int n){
  if (n < MAXN) return mu[n];
  if (map_mu[n]) return map_mu[n];
  ll res = 1;
  for (int l = 2, r; l <= n; l = r + 1){
      int t = n / l;
      r = n  / t;
      res -= 1LL * (r - l + 1) * get_mu(t);
  }
  return map_mu[n] = res;
}
\end{lstlisting}\par
\textbf{欧拉函数前缀和}$\varphi\ast 1=ID$ 
\begin{align*}
  &\sum_{i=1}^n(\varphi\ast 1)(i)=\sum_{i=1}^n1\cdot S\left(\left\lfloor\frac{n}{i}\right\rfloor\right)\\
  &\sum_{i=1}^nID(i)=\sum_{i=1}^n1\cdot S\left(\left\lfloor\frac{n}{i}\right\rfloor\right)\\
  &\frac{1}{2}n(n+1)=\sum_{i=1}^nS\left(\left\lfloor\frac{n}{i}\right\rfloor\right)\\
  &S(n)=\frac{1}{2}n(n+1)-\sum_{i=2}^nS\left(\left\lfloor\frac{n}{i}\right\rfloor\right)\\
\end{align*}
\begin{lstlisting}
ll get_phi(int n){
  if (n < MAXN) return phi[n];
  if (map_phi[n]) return map_phi[n];
  ll res = 1LL * (n + 1) * n / 2;
  for (int l = 2, r; l <= n; l = r + 1){
      int t = n / l;
      r = n / t;
      res -= 1LL * (r - l + 1) * get_phi(t);
  }
  return map_phi[n] = res;
}
\end{lstlisting}


\subsection{大数质因数分解(Pollard\_Rho算法)}
用于快速分解质因数。
\begin{lstlisting}
#include <iostream>
#include <cstdio>
#include <cstdlib>
#include <ctime>
#include <vector>
#include <algorithm>
#include <random>
using namespace std;

long long mul_mod(long long a, long long b, long long c){
    a %= c;
    b %= c;
    long long res = 0, tmp = a;
    while(b){
        if(b & 1){
            res += tmp;
            if(res > c) res -= c;
        }
        tmp <<= 1;
        if(tmp > c) tmp -= c;
        b >>= 1;
    }
    return res;
}
long long pow_mod(long long a, long long k, long long mod){
    long long base = a, res = 1;
    while(k){
        if(k & 1) res = mul_mod(res, base, mod);
        base = mul_mod(base, base, mod);
        k >>= 1;
    }
    return res;
}
bool check(long long a, long long n, long long x, long long t){
    long long res = pow_mod(a, x, n);
    long long last = res;
    for(int i = 0; i < t; i++){
        res = mul_mod(res, res, n);
        if(res == 1 && last != n - 1 && last != 1) return 1;
        last = res;
    }
    return res != 1;
}

bool Miller_Rabin(long long n){
    if(n == 2) return 1;
    if(n < 2 || n % 2 == 0) return 0;
    long long x = n - 1;
    long long t = 0;
    while((x & 1) == 0){
        x >>= 1;
        t++;
    }
    for(int i = 0; i < 10; i++){
        long long a = rand() % (n - 1) + 1;
        if(check(a, n, x, t)) return 0;
    }
    return 1;
}

vector<long long> fac;
long long gcd(long long a, long long b){
    return b == 0 ? abs(a) : gcd(b, a % b);
}

long long Pollard_rho(long long x, long long c){
    long long i = 1, k = 2;
    long long x0 = rand() % (x - 1) + 1;
    long long y = x0;
    while(1){
        i++;
        x0 = (mul_mod(x0, x0, x) + c) % x;
        long long d = gcd(y - x0, x);
        if(d != 1 && d != x) return d;
        if(y == x0) return x;
        if(i == k){
            y = x0;
            k <<= 1;
        }
    }
}
void findfac(long long n, int k){
    if(n == 1) return;
    if(Miller_Rabin(n)){
        fac.emplace_back(n);
        return;
    }
    long long p = n;
    int c = k;
    while(p >= n){
        p = Pollard_rho(p, c--);
    }
    findfac(p, k);
    findfac(n / p, k);
}
int main(){
//    srand(time(NULL));
    int t; long long n;
    scanf("%d", &t);
    while(t--){
        fac.clear();
        scanf("%lld", &n);
        findfac(n,107);
        if(fac.size() == 1) printf("Prime\n");
        else{
            sort(fac.begin(), fac.end());
            printf("%lld\n", *fac.rbegin());
        }
    }
    return 0;
}
\end{lstlisting}
