\section{BST}
\subsection{splay}
\subsection{求树中两点异或和等于其LCA的点对个数}

\begin{lstlisting}
	#include <iostream>
	#include <cstdio>
	#include <vector>
	#include <map>
	#include <set>
	using namespace std;
	const int N = 1e5 + 7;
	int a[N];
	vector<vector<int> > e;
	int sz[N], bc[N];
	int cnt[N * 10][21][2];
	int son = 0;
	long long ans = 0;
	void dfs1(int x, int fa){
		sz[x] = 1;
		for(auto v: e[x]){
			if(v == fa) continue;
			dfs1(v, x);
			sz[x] += sz[v];
			if(sz[v] > sz[bc[x]]) bc[x] = v;
		}
	}
	void update(int x, int fa, int val){
		int o = a[x] ^ val;
		if(o <= 1e6){
			for(int i = 0; i <= 20; i++){
				ans += (long long)cnt[o][i][!((x >> i) & 1)] * (1 << i);
			}
		}
		for(auto v: e[x]){
			if(v == fa) continue;
			update(v, x, val);
		}
	}
	void add(int x, int fa, int val){
		for(int i = 0; i <= 20; i++){
			cnt[a[x]][i][(x >> i) & 1] += val;
		}
		for(auto v: e[x]){
			if(v == fa) continue;
			add(v, x, val);
		}
	}
	void dfs2(int x, int fa, bool keep){
		for(auto v: e[x]){
			if(v == fa || v == bc[x]) continue;
			dfs2(v, x, 0);
		}
		if(bc[x]){
			dfs2(bc[x], x, 1);
			son = bc[x];
		}
		for(auto v: e[x]){
			if(v == fa || v == bc[x]) continue;
			update(v, x, a[x]);
			add(v, x, 1);
		}
		for(int i = 0; i <= 20; i++){
			cnt[a[x]][i][(x >> i) & 1] += 1;
		}
		son = 0;
		if(!keep){
			add(x, fa, -1);
		}
	}
	int main(){
		int n, u, v;
		scanf("%d", &n);
		e.resize(n + 1);
		for(int i = 1; i <= n; i++){
			scanf("%d", &a[i]);
		}
		for(int i = 1; i < n; i++){
			scanf("%d%d", &u, &v);
			e[u].emplace_back(v);
			e[v].emplace_back(u);
		}
		dfs1(1, 0);
		dfs2(1, 0, 0);
		printf("%lld\n", ans);
		return 0;
	}
\end{lstlisting}

\subsection{求满足以下条件的点对(x, y)个数}
\begin{itemize}
	\item $x \neq y$
	\item $LCA(x, y) \neq x, LCA(x, y) \neq y$
	\item $x, y$ 距离不超过 $k$ 
	\item $v_x + v_y = 2v_{LCA(x,y)}$
\end{itemize}
\begin{lstlisting}
	#include <iostream>
	#include <cstdio>
	#include <vector>
	using namespace std;
	const int N = 1e5 + 7;
	vector<vector<int> > e;
	int n, k;
	int w[N];
	int cnt;
	int rt[N];
	struct NODE{
		int ls, rs, sum;
	}t[N * 40];
	int dep[N], sze[N], bc[N];
	long long ans = 0;
	void pre(int x, int fa){
		dep[x] = dep[fa] + 1;
		sze[x] = 1;
		for(auto v: e[x]){
			pre(v, x);
			if(sze[v] > sze[bc[x]]) bc[x] = v;
			sze[x] += sze[v];
		}
	}
	void update(int &o, int l, int r, int x, int val){
		if(o == 0) o = ++cnt;
		t[o].sum += val;
		if(l == r) return;
		int mid = (l + r) >> 1;
		if(x <= mid) update(t[o].ls, l, mid, x, val);
		else update(t[o].rs, mid + 1, r, x, val);
	}
	int query(int o, int l, int r, int ql, int qr){
		if(o == 0) return 0;
		if(qr < l || ql > r) return 0;
		if(ql <= l && r <= qr) return t[o].sum;
		int mid = (l + r) >> 1;
		int res = 0;
		if(ql <= mid) res += query(t[o].ls, l, mid, ql, qr);
		if(qr > mid) res += query(t[o].rs, mid + 1, r, ql, qr);
		return res;
	}
	void cal(int x, int fa, int f){
		int r = w[f] * 2 - w[x];
		if(r <= n){
			ans += query(rt[r], 1, n, dep[f] + 1, min(n, k - (dep[x] - 2 * dep[f])));
		}
		for(auto v: e[x]){
			cal(v, x, f);
		}
	}
	void add(int x, int fa, int val){
		update(rt[w[x]], 1, n, dep[x], val);
		for(auto v: e[x]){
			add(v, x, val);
		}
	}
	void dfs(int x, int fa, bool keep){
		for(auto v: e[x]){
			if(v == bc[x]) continue;
			dfs(v, x, 0);
		}
		if(bc[x]) dfs(bc[x], x, 1);
		for(auto v: e[x]){
			if(v == bc[x]) continue;
			cal(v, x, x);
			add(v, x, 1);
		}
		update(rt[w[x]], 1, n, dep[x], 1);
		if(!keep){
			add(x, fa, -1);
		}
	}
	int main(){
		int u, v;
		scanf("%d%d", &n, &k);
		e.resize(n + 1);
		for(int i = 1; i <= n; i++){
			scanf("%d", &w[i]);
		}
		for(int i = 2; i <= n; i++){
			scanf("%d", &u);
			e[u].emplace_back(i);
		}
		pre(1, 0);
		dfs(1, 0, 1);
		printf("%lld\n", ans * 2);
		return 0;
	}
\end{lstlisting}