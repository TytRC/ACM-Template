\subsection{可持久化Trie树}
\textbf{求点权树最大异或路径}
\begin{lstlisting}
/*
   给定一棵N个点的带权树,结点下标从1开始到N。寻找树中找两个结点,求最长的异或路径。
   异或路径指的是指两个结点之间唯一路径上的所有边权的异或。
*/
#include<iostream>
#include<cstdio>
using namespace std;
int trie[5000000][2],cnt;
struct TU{
	int to,nxt,w;
}t[200005];
int ecnt,last[100005],ans=0;
void add(int u,int v,int w){ t[++ecnt].to=v; t[ecnt].nxt=last[u]; t[ecnt].w=w; last[u]=ecnt; }
void insert(int x){
	int i,u=0,v;
	for(i=30;i>=0;i--){
		v=(1<<i)&x?1:0;
		if(trie[u][v]==0) trie[u][v]=++cnt;
		u=trie[u][v];
	}
}
int query(int x){
	int res=0,u=0,i,v;
	for(i=30;i>=0;i--){
		v=x&(1<<i)?0:1;
		if(trie[u][v]){res|=1<<i; u=trie[u][v];}
		else u=trie[u][1-v];
	}
	return res;
}
void dfs(int fa,int u,int nw){
	int e=last[u];
	while(e){
		int v=t[e].to;
		if(v==fa) {e=t[e].nxt;continue;}
		int yw=nw^t[e].w;
		ans=max(ans,query(yw));
		insert(yw); dfs(u,v,yw);
		e=t[e].nxt;
	}
}
int main(){
	int i,j,k,m,n,u,v,w;
	scanf("%d",&n);
	for(i=1;i<n;i++){
		scanf("%d%d%d",&u,&v,&w);
		add(u,v,w);add(v,u,w);
	}
	insert(0);
	dfs(0,1,0);
	printf("%d\n",ans);
	return 0;
}
\end{lstlisting}

\textbf{现在有一颗以 1 为根节点的由 n 个节点组成的树,节点从 1 至 n 编号。树上每个节点上都有一个权值 $v_i$。现在有 q 次操作,操作如下:}
\begin{itemize}
    \item \textit{1 x z}:查询节点 x 的子树中的节点权值与 z 异或结果的最大值。
    \item \textit{2 x y z}:查询节点 x 到节点 y 的简单路径上的节点的权值与 z 异或结果最大值。
\end{itemize}
\begin{lstlisting}
#include<iostream>
#include<cstdio>
using namespace std;
const int N=1e5+7;
struct Graph{
	int nxt,to;
}t[N<<1];
int val[N];
int last[N],ecnt;
void build(int u,int v){
	t[++ecnt].to=v;
	t[ecnt].nxt=last[u];
	last[u]=ecnt;
}
int dep[N],f[N][31],bl[N],br[N],ct,lg[N];
int trie[N*70][2],rt1[N],rt2[N],sze[N*70];
int cnt;
int clone(int o){
	int r=++cnt;
	trie[r][0]=trie[o][0];
	trie[r][1]=trie[o][1];
	sze[r]=sze[o];
	return r;
}
void inst(int &o,int x,int b){
	o=clone(o);sze[o]++;
	if(b==-1) return;
	int v=(x>>b)&1;
	inst(trie[o][v],x,b-1);
}
void dfs(int fa,int u){
	bl[u]=++ct;dep[u]=dep[fa]+1;
	f[u][0]=fa;
	int i;
	for(i=1;i<=17;i++){
		f[u][i]=f[f[u][i-1]][i-1];
	}
	rt2[u]=rt2[fa];
	inst(rt2[u],val[u],30);
	rt1[ct]=rt1[ct-1];
	inst(rt1[ct],val[u],30);
	int e=last[u];
	while(e){
		int v=t[e].to;
		if(v==fa){
			e=t[e].nxt;
			continue;
		}
		dfs(u,v);
		e=t[e].nxt;
	}
	br[u]=ct;
}
int lca(int x,int y){
	int i;
	if(dep[x]<dep[y]){
		int tmp=x;x=y;y=tmp;
	}
	while(dep[x]!=dep[y]){
		x=f[x][lg[dep[x]-dep[y]]];
	}
	if(x==y) return x;
	for(i=17;i>=0;i--){
		if(f[x][i]!=f[y][i]){
			x=f[x][i];
			y=f[y][i];
		}
	}
	return f[x][0];
}
int query1(int o,int x){
	int res=0;
	int i,ul=rt1[bl[o]-1],ur=rt1[br[o]],v;
	for(i=30;i>=0;i--){
		v=((x>>i)&1)^1;
		if(sze[trie[ur][v]]-sze[trie[ul][v]]){
			res|=1<<i;
			ur=trie[ur][v];
			ul=trie[ul][v];
		}
		else{
			ur=trie[ur][v^1];
			ul=trie[ul][v^1];
		}
	}
	return res;
}
int query2(int ox,int oy,int x){
	int res=0;
	int i,ol=lca(ox,oy),v;
	int ux=rt2[ox],uy=rt2[oy],ul=rt2[ol],uf=rt2[f[ol][0]];
	for(i=30;i>=0;i--){
		v=((x>>i)&1)^1;
		if(sze[trie[ux][v]]+sze[trie[uy][v]]-sze[trie[ul][v]]-sze[trie[uf][v]]){
			res|=1<<i;
		}
		else{
			v^=1;
		}
		ux=trie[ux][v];
		uy=trie[uy][v];
		ul=trie[ul][v];
		uf=trie[uf][v];
	}
	return res;
}
int main(){
	int i,j,k,m,n,q,u,v,opt,x,y,z;
	scanf("%d%d",&n,&q);
	for(i=2;i<=n;i++) lg[i]=lg[i>>1]+1;
	for(i=1;i<=n;i++) scanf("%d",&val[i]);
	for(i=1;i<n;i++){
		scanf("%d%d",&u,&v);
		build(u,v);build(v,u);
	}
	dfs(0,1);
	for(i=1;i<=q;i++){
		scanf("%d",&opt);
		if(opt==1){
			scanf("%d%d",&x,&z);
			printf("%d\n",query1(x,z));
		}
		else{
			scanf("%d%d%d",&x,&y,&z);
			printf("%d\n",query2(x,y,z));
		}
	}
	return 0; 
}
\end{lstlisting}