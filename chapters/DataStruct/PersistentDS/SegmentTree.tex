\subsection{可持久化线段树}
\subsubsection{求区间第k小(主席树)}
\begin{lstlisting}
/*
给定 n 个整数构成的序列 a,将对于指定的闭区间 [l,r] 查询其区间内的第 k 小值。
*/
#include<iostream>
#include<cstdio>
#include<algorithm>
using namespace std; 
const int N=2e5+7;
struct PST{
	int ls,rs;
	int num;
}t[N*20];
int a[N],b[N];
int rt[N],cnt;
void pushup(int o){
	t[o].num=t[t[o].ls].num+t[t[o].rs].num;
}
void build(int &o,int l,int r){
	o=++cnt;
	if(l==r){
		t[o].num=0;
		return;
	}
	int mid=(l+r)>>1;
	build(t[o].ls,l,mid);build(t[o].rs,mid+1,r);
	pushup(o);
}
int clone(int x){
	int r=++cnt;
	t[r]=t[x];
	return r;
}
void update(int &o,int l,int r,int c){
	o=clone(o);
	if(l==r){
		t[o].num++;
		return;
	}
	int mid=(l+r)>>1;
	if(c<=mid) update(t[o].ls,l,mid,c);
	else update(t[o].rs,mid+1,r,c);
	pushup(o);
}
int query(int ox,int oy,int l,int r,int k){
	if(l>=r) return l;
	int mid=(l+r)>>1;
	int z=t[t[oy].ls].num-t[t[ox].ls].num;
	if(z>=k) return query(t[ox].ls,t[oy].ls,l,mid,k);
	else return query(t[ox].rs,t[oy].rs,mid+1,r,k-z);
}
int main(){
	int i,j,k,m,n,x,y;
	scanf("%d%d",&n,&m);
	for(i=1;i<=n;i++){
		scanf("%d",&a[i]);
		b[i]=a[i];
	}
	sort(b+1,b+n+1);
	int d=unique(b+1,b+n+1)-b-1;
	build(rt[0],1,d);
	for(i=1;i<=n;i++){
		x=lower_bound(b+1,b+d+1,a[i])-b;
		rt[i]=rt[i-1];
		update(rt[i],1,d,x);
	}
	for(i=1;i<=m;i++){
		scanf("%d%d%d",&x,&y,&k);
		printf("%d\n",b[query(rt[x-1],rt[y],1,d,k)]);
	}
	return 0;
}
\end{lstlisting}


\subsubsection{树上路径点权第 k 小}
\begin{lstlisting}
#include<iostream>
#include<cstdio>
#include<algorithm>
#define lson t[o].ls
#define rson t[o].rs
using namespace std;
const int N=1e5+7;
int a[N],b[N];
struct Graph{
	int to,nxt;
}e[N<<1];
int last[N],ecnt,f[N][20];
void addEdge(int u,int v){
	e[++ecnt].to=v;
	e[ecnt].nxt=last[u];
	last[u]=ecnt;
}
struct PST{
	int ls,rs,num;
}t[N*30];
int cnt,d;
int rt[N];
int dep[N];
int lg[N];
void pushup(int o){
	t[o].num=t[lson].num+t[rson].num;
}
void build(int &o,int l,int r){
	o=++cnt;
	if(l==r){
		t[o].num=0;
		return;
	}
	int mid=(l+r)>>1;
	build(lson,l,mid);
	build(rson,mid+1,r);
}
int clone(int x){
	int r=++cnt;
	t[r]=t[x];
	return r;
}
void update(int &o,int l,int r,int c){
	o=clone(o);
	if(l==r){
		t[o].num++;
		return;
	}
	int mid=(l+r)>>1;
	if(c<=mid) update(lson,l,mid,c);
	else update(rson,mid+1,r,c);
	pushup(o);
}
void dfs(int fa,int x){
	f[x][0]=fa;
	for(int i=1;i<=18;i++) f[x][i]=f[f[x][i-1]][i-1];
	dep[x]=dep[fa]+1;
	rt[x]=rt[fa];
	update(rt[x],1,d,lower_bound(b+1,b+d+1,a[x])-b);
	int g=last[x];
	while(g){
		int v=e[g].to;
		if(v!=fa){
			dfs(x,v);
		}
		g=e[g].nxt;
	}
}
int lca(int u,int v){
	if(dep[u]<dep[v]){
		int tmp=u;u=v;v=tmp;
	}
	while(dep[u]!=dep[v]){
		u=f[u][lg[dep[u]-dep[v]]];
	}
	if(u==v) return u;
	for(int i=18;i>=0;i--){
		if(f[u][i]!=f[v][i]){
			u=f[u][i];
			v=f[v][i];
		}
	}
	return f[u][0];
}
int query(int ox,int oy,int oz,int of,int l,int r,int k){
	if(l==r) return l;
	int p=t[t[ox].ls].num+t[t[oy].ls].num-t[t[oz].ls].num-t[t[of].ls].num;
	int mid=(l+r)>>1;
	if(p>=k) return query(t[ox].ls,t[oy].ls,t[oz].ls,t[of].ls,l,mid,k);
	else return query(t[ox].rs,t[oy].rs,t[oz].rs,t[of].rs,mid+1,r,k-p);
}
int main(){
	int i,j,k,m,n,x,y,ls=0;
	//freopen("test.in","r",stdin);
	scanf("%d%d",&n,&m);
	for(i=2;i<=n;i++) lg[i]=lg[i>>1]+1;
	for(i=1;i<=n;i++){
		scanf("%d",&a[i]);
		b[i]=a[i];
	}
	sort(b+1,b+n+1);
	d=unique(b+1,b+n+1)-b-1;
	build(rt[0],1,d);
	for(i=1;i<n;i++){
		scanf("%d%d",&x,&y);
		addEdge(x,y);addEdge(y,x);
	}
	dfs(0,1);
	for(i=1;i<=m;i++){
		scanf("%d%d%d",&x,&y,&k);
		x=x^ls;
		int la=lca(x,y);
		ls=b[query(rt[x],rt[y],rt[la],rt[f[la][0]],1,d,k)];
		printf("%d\n",ls);
	}
	return 0;
}
\end{lstlisting}