\subsubsection{可持久化并查集}
给定 $n$ 个集合,第 $i$ 个集合内初始状态下只有一个数,为 $i$ 。
有 $m$ 次操作。操作分为 $3$ 种:
\begin{itemize}
    \item \textit{1 a b}:合并 a,b 所在集合
    \item \textit{2 k}:回到第 k 次操作(执行三种操作中的任意一种都记为一次操作)之后的状态
    \item \textit{3 a b}:询问 a,b 是否属于同一集合,如果是则输出 1 ,否则输出 0。
\end{itemize}
\begin{lstlisting}
#include <iostream>
#include <cstdio>
using namespace std;
const int N = 1e5 + 7;
int cnt = 0;
struct ZXS{
    int lson, rson, fa, dep;
}t[N * 40];
int rt[2 * N];
int n;
void build(int &o, int l, int r){
    o = ++ cnt;
    if(l == r){
        t[o].fa = l;
        return;
    }
    int mid = (l + r) >> 1;
    build(t[o].lson, l, mid);
    build(t[o].rson, mid + 1, r);
}
int clone(int o){
    int y = ++cnt;
    t[y] = t[o];
    return y;
}
void update(int &o, int l, int r, int x, int y){
    o = clone(o);
    if(l == r){
        t[o].fa = y;
        return;
    }
    int mid = (l + r) >> 1;
    if(x <= mid) update(t[o].lson, l, mid, x, y);
    else update(t[o].rson, mid + 1, r, x, y);
}
void add(int &o, int l, int r, int x){
    if(l == r){
        t[o].dep++;
        return;
    }
    int mid = (l + r) >> 1;
    if(x <= mid) add(t[o].lson, l, mid, x);
    else add(t[o].rson, mid + 1, r, x);
}
int query_fa(int o, int l, int r, int x){
    if(l == r){
        return o;
    }
    int mid = (l + r) >> 1;
    if(x <= mid) return query_fa(t[o].lson, l, mid, x);
    else return query_fa(t[o].rson, mid + 1, r, x);
}
int find(int x, int nw){
    int y = query_fa(rt[nw], 1, n, x);
    while(x != t[y].fa){
        x = t[y].fa;
        y = query_fa(rt[nw], 1, n, x);
    }
    return y;
}
int main(){
    int m, opt, x, y;
    scanf("%d%d", &n, &m);
    build(rt[0], 1, n);
    for(int i = 1; i <= m; i++){
        scanf("%d", &opt);
        rt[i] = rt[i - 1];
        if(opt == 1){
            scanf("%d%d", &x, &y);
            int xx = find(x, i), yy = find(y, i);
            if(t[xx].dep > t[yy].dep) swap(xx, yy);
            if(xx == yy) continue;
            update(rt[i], 1, n, t[xx].fa, t[yy].fa);
            if(t[xx].dep == t[yy].dep){
                int tmp = rt[i];
                add(tmp, 1, n, t[yy].fa);
            }
        }
        else if(opt == 2){
            scanf("%d", &x);
            rt[i] = rt[x];
        }
        else if(opt == 3){
            scanf("%d%d", &x, &y);
            if(t[find(x, i)].fa == t[find(y, i)].fa){
				printf("1\n");
			}
            else{
            	printf("0\n");
			}
        }
    }
    return 0;
}
\end{lstlisting}